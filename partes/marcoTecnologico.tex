\chapter{Marco técnologico}
\label{capituloTecnologico}

En este capítulo se dará una breve explicación de las herramientas utilizadas para implementar y probar el sistema implementado. De igual manera, se detallarán las especificaciones de la estación de trabaja que se utilizó para realizar dicho trabajo.  

\section{Herramientas utilizadas}

En esta sección se explicarán las herramientas utilizadas al momento de implementar el sistema  de detección de intrusiones.

En primer lugar se detallarán las caracteristicas del sistema de monitorización de red, (\textit{Bro}) cuyo lenguaje de programación fué utilizado para implementar el IDS. Luego, se explicará cual fué el controlador de versiones utilizado en el proyecto (\textit{GitHub}). Finalmente se describirá la herramienta utilizada para realizar las capturas de paquetes de red que fueron utilizadas para efectuar las pruebas sobre el sistema (\textit{Wireshark}).  

\subsection{Bro}\label{ssec:Bro}

Bro es un software ``open source'' capáz de analizar a detalle las actividades que ocurren en la red. Se utiliza principalmente para inspeccionar todo el tráfico y detectar signos de actividad sospechosa. Sin embargo, Bro soporta una amplia gama de tareas de análisis de tráfico incluso fuera del dominio de seguridad, incluyendo mediciones de rendimiento y ``trouble-shouting''.\cite{Bro}

Este software, posee un conjunto extenso de ``logs'' que registran las actividades que se observan en la red. Estos ``logs'' almacenan información de la capa de aplicación, como por ejemplo, los URIs y las respuestas del servidor en una sesión de tipo HTTP, o las solicitudes y respuestas DNS en una sesión DNS. Bro escribe todos estos datos en archivos con formatos bien predeterminados con la intención de facilitar el procesamiento de la información almacenada por programas externos. No obstante, el usuario tiene la posibilidad de elegir el formato en el que se almacenan los datos.

Por otra parte, Bro incluye también un lenguaje de ``scripting'', Turing completo y orientado a eventos, cuya función es extender y personalizar las funcionalidades primarias que otorga el mismo. Este lenguaje de ``scripting'' posee una biblioteca estándar. 

Así mismo, el lenguaje de Bro posee diversos ``frameworks'' que facilitan el trabajo al momento de programar nuevas funcionalidades. Los ``frameworks'' utilizados en el sistema implementado fueron: el ``Input Framework'' y el ``Logging Framework''. El ``Input Framework'' permite leer de manera mas cómoda y eficiente la información que se encuentra dentro de los ``logs'' del sistema. Este ``framework'' posee dos modalidades. La primera modalidad consiste en leer la información que se encuentra dentro de los ``logs'' para luego almacenarla en una estructura de datos de tipo ``table''. La segunda modalidad lee la información y la envía a un evento anteriormente programado.Por otra parte, el ``Logging Framework'', tiene como tarea facilitar el manejo y la escritura de los ``logs''. Este, permite crear, escribir datos de forma organizada y filtrar información de los ``logs''.

Como se mencionó con anterioridad, el lenguaje de ``scripting'' de Bro es un lenguaje orientado a objetos. Sin embargo, es importante recalcar, que el núcleo o motor principal de Bro es el motor de eventos, el cual reduce la información entrante de la red, en eventos de alto nivel. Por ejemplo, si se recibe una solicitud de tipo HTTP, Bro se encargará de convertir la misma en un evento de tipo ``http\_request'' en donde se almacenará información como: la dirección IP, los puertos involucrados, el URI y la versión de tipo HTTP en uso.

A su vez, los eventos creados por Bro a partir de la información entrante de la red son enviados a los manejadores de eventos escritos en el lenguaje de ``scripting'' de Bro. Estos manejadores administran dichos eventos a tráves de una cola ordenada.

En Bro hay eventos primitivos, como es el caso de ``http\_request''  --evento generado por las solicitudes tipo HTTP -- , ``dns\_request'', --evento generado por las solicitudes tipo DNS-- , ``bro\_init'' -- evento generado al momento en el que Bro inicializa-- y ``bro\_done'' --evento que se genera cuando se culminan de realizar todas las tareas que se deben realizar--. Sin embargo, este software brinda la capacidad a los programadores de implementar eventos propios. Estos eventos son un tipo especial de función que poseen las características de: no retornar valores, poder ser programados para una ejecución posterior y poseer una prioridad asociada y configurable que determine el orden en el que serán ejecutados.

Así como se pueden escribir eventos en Bro, también se pueden implementar funciones haciendo uso de la palabra reservada ``function''.

Por otra parte, el lenguaje de Bro además de permitir implementar eventos y funciones, posee diversos tipos y estructura de datos que facilitan el manejo de la información. 
El conjunto de tipos de datos que posee el mismo se basa en los tipos de datos comunes que encontramos en la mayoría de los lenguajes de programación como: el ``int'', que representa los números enteros de 64 bits, ``double'' que representa los números flotantes de 64 bits, ``bool'' para los booleanos y ``count''  para los números enteros sin signo de 64 bits, y en un grupo de tipos de datos  mas especifico como: el ``addr'' que representa direcciones ip, ``port'' para los puertos de la capa de transporte, ``subnet'' para las mascaras de subred, ``time'' para almacenar datos que representen tiempo, ``interval'' para representar intervalos de tiempo y ``pattern'' para las expresiones regulares.

Las estructura de datos que permite el lenguaje son: los conjuntos (``set''), las tablas (``table''), vectores (``vector'') y registros (``record'').

\subsection{GitHub}


GitHub es un controlador de versiones basado en Git y un servidor de ``hosting''. Este ofrece todas las funcionalidades de Git y a su vez, añade sus propias funcionalidades. GitHub provee control de acceso y varias caracteristicas para la colaboración de proyectos como: el seguimiento de errores, manejador de tareas y wikis. \cite{GitHub}

Por otra parte, Git es un programa ``open source'' para el rastreo de cambios en archivos de texto. Esta herramiento fué escrita por el autor del sistema operativo Linux. \cite{Git}

\subsection{Wireshark}

Wireshark el analizador de paquetes de red ``open source''. Este permite  observar lo que esta ocurriendo en la red a un nivel microscopico \cite{wireshark1}.

Un analizador de paquetes de red captura los paquetes de red e intenta mostrar los datos del mismo de la manera mas detallada posible \cite{wireshark2}.

Algunos usos que le dan las personas a Wireshark son:

\begin{itemize}
\item Los administradores de red lo usan para solucionar problemas en la red.
\item Los ingenieros en seguridad lo utilizan para examinar problemas de seguridad.
\item Los desarrolladores lo utilizan para encontrar errores en los protocolos implementados
\item Las personas lo utilizan para aprender protocolos de red.
\end{itemize}

\section{Estación de trabajo}

\subsection{Especificaciones}
A continuación, se describirán las especificaciones de la máquina que se utilizó en el proyecto.

\begin{itemize}
\item \textbf{Procesador}: 7th Generation Intel® Core™ i3 processor.
\item \textbf{Memoria (RAM)}: 4GB
\item \textbf{Sistema operativo}: Ubuntu 14.04
\item \textbf{HDD (principal)}: 500GB
\end{itemize}
