\setcounter{page}{5}
\begin{center}

{\bfseries IMPLEMENTACIÓN DE UNA APLICACIÓN QUE VALIDA EL RESPALDO DE MENSAJES EN CAJAS NEGRAS DE AVIONES AIRBUS\\}
\bigskip
Por: \\ Pablo Ricardo Maldonado Montilva\\
\bigskip
\bigskip
{\bf RESUMEN} \pdfbookmark[0]{Resumen}{resumen} % Sets a PDF bookmark for the dedication
\end{center}	

El proyecto consistió en la implementación de una aplicación de escritorio que valida el respaldo de mensajes en los registradores de vuelo\footnote{mejor conocidos como ``las cajas negras"} de los aviones del grupo industrial europeo Airbus en Toulouse, Francia. Para ejecutar esta tarea, es necesario realizar la comparación de los datos respaldados con aquellos recibidos por los controladores aéreos en tierra. Este producto fue concebido por el equipo de pruebas COM/ATM (Comunicación y Gestión del Tráfico Aéreo, por sus siglas en inglés) de la empresa SOGETI High Tech que presta servicios como subcontratista de la filial Airbus S.A.S (Sociedad por Acciones Simplificada, por sus siglas en francés). Los objetivos de la herramienta son reducir el tiempo de validación de los sistemas de comunicación durante las jornadas de pruebas, y evitar los posibles errores humanos al realizar esta actividad de forma manual. La metodología utilizada fue una adaptación del método de desarrollo ágil Scrum. Durante el proyecto, se realizaron las siguientes actividades: familiarización con el contexto de los sistemas de comunicación aeronáuticos, levantamiento de requerimientos, definición de la arquitectura del producto, implementación de funcionalidades y ejecución de pruebas.  