\setcounter{page}{5}
\begin{center}

{\bfseries MÓDULO PARA LA DETECCIÓN DE
ATAQUE MEDIANTE BRO\\}
\bigskip
Por: \\ Alejandra Del Carmen Cordero García\\
\bigskip
\bigskip
{\bf RESUMEN} \pdfbookmark[0]{Resumen}{resumen} % Sets a PDF bookmark for the dedication
\end{center}	

La pasantía larga consistió en desarrollar e implementar los
módulos necesarios para construir un detector de intrusiones para servicios
web, basado en SSM (por el ingles, ``Segmented Stochastic Modelling''),
que se pueda utilizar en redes de explotación. El detector fué implementado
haciendo uso del lenguaje de ``scripting'' de un sistema de monitorización
llamado Bro. Este sistema, se encargará de tomar peticiones HTTP de tipo
GET, extraer sus URIs y analizar si los mismas poseen anomalías de una
manera probabilística, utilizando modelos de normalidad. En el caso en el
que se perciba alguna anormalidad, el sistema se encargará de emitir una
alarma.
El proyecto estuvo dividido en varias fases: La primera fase consistió en
estudiar SSM y sus conceptos relacionados, así como familiarizarse con la
herramienta Bro; la segunda fase se basó en diseñar y dividir las tareas del
sistema en tres módulos principales: un modulo de segmentación, un módulo
de evaluación y un módulo de entrenamiento; en la tercera fase se realizó la
implementación y las pruebas funcionales del detector de intrusiones; en la
cuarta fase se realizaron las pruebas operativas al sistema completo haciendo
uso de bases de datos con trazas normales y bases de datos con trazas
anómalas; y finalmente,en la sexta fase, se estructuró y se redactó la memoria
del proyecto.

% Las palabras clave son generalmente los nombres de áreas de investigación a
% los cuales está asociado el trabajo. Generalmente son tres o cuatro.
\vfill
\textbf{Palabras clave}: SSM, HTTP, GET, URI, Bro.
 