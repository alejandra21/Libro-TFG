\chapter*{Conclusiones y Recomendaciones}
\label{conclusiones}
\lhead{\emph{Conclusiones y Recomendaciones}}
% Incluir recomendaciones para trabajos futuros
\addcontentsline{toc}{chapter}{Conclusiones}

En este capítulo se presentarán las conclusiones y los aportes realizados por este proyecto. Así mismo, se mostrará una lista de recomendaciones, que expone un conjunto de ideas de futuras mejoras que se pueden aplicar al proyecto.

\section*{Conclusiones}

En el presente proyecto se realizaron todos los objetivos planteados de manera satisfactoria. En primer lugar, se realizó un estudio de SSM (``Stochastic Structural Model'') \cite{ssm} y todos sus conceptos asociados. Una vez realizado este investigación, se diseñó una arquitectura modular del sistema, en la cual se propuso la construcción de tres módulos: uno para capturar, filtrar y segmentar los paquetes de tipo \textit{GET} del protocolo \textit{HTTP}; otro módulo que permite evaluar el índice de anormalidad de los URI y un tercer módulo para estimar los modelos de normalidad. Luego de modelar la arquitectura del sistema, se implementó cada una de las funciones del mismo haciendo uso de la herramientas y del lenguaje de ``scripting'' de Bro, con la intención de obtener un IDS basado en SSM que puede ser utilizado por servicios web en producción. Por último, se realizaron pruebas tanto funcionales como operativas. 

Los aportes realizados por este proyecto son los siguientes:

\begin{enumerate}
\item Se ha implementado un sistema que realiza detección de intrusiones en los servidores web.
\item Se ha implementado un sistema con tres funcionalidades operativas: una para realizar detección de intrusiones y dos para realizar modelos de normalidad de los servicios web.
\item Se han construido modelos de normalidad de servicios web que podrían ser utilizados para la detección de intrusiones.
\end{enumerate}

\section*{Recomendaciones}

\begin{enumerate}
\item Calcular de manera experimental los valores óptimos de $\theta$ y los de probabilidad fuera de vocabulario que generen la menor cantidad posible de falsos positivos, haciendo uso de bases de datos mas extensas.
\item Agregarle al IDS un conjunto de firmas de ataques conocidos aplicando la técnica propuesta en \cite{firmas}.
\item Probar el sistema en un entorno de producción.
\item Ampliar el sistema para que funcione en otros protocolos de la red.
\end{enumerate}
