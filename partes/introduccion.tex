\chapter*{INTRODUCCIÓN}
\label{intro}
\lhead{\emph{Introducción}}
\addcontentsline{toc}{chapter}{Introducción}

% Descripción del problema, de lo general hacia lo específico

Hoy en dia es muy difícil imaginar el mundo sin el Internet. Esta gran red de
redes se ha convertido en una parte importante en la comunicación en el mundo.
Un gran porcentaje del uso que se le da al Internet proviene de las aplicaciones
web. Para compartir toda la información brindada por estas aplicaciones a través del Internet, se hace uso del protocolo HTTP ("Hypertext Transfer Protocol"), un protocolo cliente-servidor. Los servidores HTTP, son los encargados de suplir las peticiones de sus clientes y enviar o recibir la información por los mismos.
Este tipo de servidores son un punto apetecible para los hackers ya que los
mismos contienen informacion importante de las aplicaciones web. Como acceso a
la base de datos de la aplicación que contiene las claves, los nombres de usuarios
y número de tarjetas de crédito. Por esta razón, es de total importancia velar por su seguridad.

Este punto será el tema central en el presente trabajo. En el
mismo, explicará la implementación y las bases téoricas de un protocolo desarrollado en
la Universidad de Granada llamado SSM ("Segmented Stochastic Modelling") en un
lenguaje se "scripting" del analizador de redes Bro.

El objetivo de implementar este sistema es el de crear un detector de instrusiones
(IDS) que detecte de manera adecuada intrusiones en servidores de tipo
HTTP.

La estructura del siguiente trabajo será la siguiente:
En el capítulo uno se hablará sobre el planteamiento del problema,
los objetivos tanto generales como especificos y la planificación que se utilizó para
realizar el proyecto.
Por otra parte,en el segundo capítulo se tocarán temas relacionados con el
estado del arte del proyecto. En primero lugar se explicará la definición de IDS y
los tipos de IDS que existen, se explicará la composición de los URI según el RFC
(mencionar el numero del RFC) y se dará una descripción del funcionamiento SSM
los modelos que lo conforman. Además, se describirá la herramiento utilizada
y el lenguaje de "scripting" que se utilizó para implementar el sistema.
En el tercer capítulo se hablará sobre el diseño del sistema. Aquí se describirán
los módulos en los que esta dividido el IDS y como es el funcionamiento de cada
uno. Los detalles de implementacion de estos se explicarán en el capítulo cuarto.
El quinto capítulo tocará el tema de la evaluación y las pruebas. En este capítulo
se hablará sobre la base de datos utilizada para realizar las pruebas y los
resultados arrojados por las mismas. Además se describirán un poco las pruebas
tanto operativas como funcionales realizadas sobre el sistema.

% Trabajos anteriores

% Objetivo general

% Objetivos específicos

% Organización del trabajo

% Se describe brevemente qué se hace en cada capítulo


