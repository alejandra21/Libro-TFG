\chapter*{Introducción}
\label{intro}
\lhead{\emph{Introducción}}
\addcontentsline{toc}{chapter}{Introducción}

% Descripción del problema, de lo general hacia lo específico



Este Trabajo Final de Grado consistió en desarrollar e implementar los módulos necesarios para implementar un detector de intrusiones para servicios web, basado en SSM (por el ingles,``Segmented Stochastic Modelling'')\cite{ssm}, efectivo y que se pueda utilizar en redes de explotación. El detector fué implementado haciendo uso del lenguaje de ``scripting'' de un sistema de monitorización llamado Bro. Este sistema, se encargará de tomar peticiones HTTP de tipo GET, extraer sus URIs y analizar si los mismas poseen anomalías de una manera probabilística, utilizando modelos de normalidad. En el caso en el que se perciba alguna anormalidad, el sistema se encargará de emitir una alarma.
 
El proyecto estuvo dividido en varias fases: La primera fase consistió en estudiar SSM y sus conceptos relacionados, así como familiarizarse con la herramienta Bro; la segunda fase se basó en diseñar y dividir las tareas del sistema en tres módulos principales: un modulo de segmentación, un modulo de evaluación y un modulo de entrenamiento; en la tercera fase se realizó la implementación y las pruebas funcionales del detector de intrusiones; en la cuarta fase se realizaron las pruebas operativas al sistema completo haciendo uso de bases de datos con trazas normales y bases de datos con trazas anómalas; y finalmente,en la sexta fase, se estructuró y se redactó la memoria del proyecto.

% Trabajos anteriores

% Objetivo general

% Objetivos específicos

% Organización del trabajo

% Se describe brevemente qué se hace en cada capítulo


