\chapter*{Introducción}
\label{intro}
\lhead{\emph{Introducción}}
\addcontentsline{toc}{chapter}{Introducción}

% Descripción del problema, de lo general hacia lo específico

Hoy en día es muy difícil imaginar el mundo sin el Internet. Esta gran red de
redes se ha convertido en una parte importante en la comunicación en el mundo.
Un gran porcentaje del uso que se le da al Internet proviene de las aplicaciones
web. Para compartir toda la información brindada por estas aplicaciones a través del Internet, se hace uso del protocolo HTTP (``Hypertext Transfer Protocol''), un protocolo cliente-servidor. Los servidores HTTP, son los encargados de suplir las peticiones de sus clientes y enviar o recibir la información a los mismos.

Este tipo de servidores son un punto apetecible para los hackers ya que contienen información importante de las aplicaciones web. Por esta razón, es de total importancia velar por su seguridad.

Este punto será el tema central en el presente trabajo. En el
mismo, se explicará la implementación y las bases téoricas de un protocolo desarrollado en
la Universidad de Granada llamado SSM (``Segmented Stochastic Modelling'') en un
lenguaje se ``scripting'' del analizador de redes, Bro.

El objetivo de implementar este sistema es el de crear un detector de intrusiones
(IDS) que detecte de manera adecuada intrusiones en servidores de tipo
HTTP.

La estructura del siguiente trabajo será la siguiente:

\begin{itemize}
\item En el primer capítulo se hablará sobre el planteamiento del problema, la justificación e importancia, 
los objetivos tanto generales como específicos y la planificación que se utilizó para
realizar el proyecto.
\item En el segundo capítulo se describirá el lugar donde se realizó la pasantía, es decir, la Universidad de Granada
\item En el tercer capítulo se explicará la definición de IDS y
los tipos de IDS que existen, la composición de los URI y se dará una descripción del funcionamiento SSM y los modelos que lo conforman
\item En el cuarto capítulo se describirá la herramienta utilizada
y el lenguaje de ``scripting'' que se utilizó para implementar el sistema
\item En el quinto capítulo se explicará el diseño del sistema, los detalles de implementación, las bases de datos utilizada para realizar las pruebas y los resultados arrojados por las mismas
\item En el sexto capítulo se mostrarán las conclusiones del presente trabajo.
\end{itemize}

% Trabajos anteriores

% Objetivo general

% Objetivos específicos

% Organización del trabajo

% Se describe brevemente qué se hace en cada capítulo


