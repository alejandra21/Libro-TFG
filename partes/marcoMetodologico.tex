\chapter{Marco Metodológico}
\label{capituloMetodologico}

En este capítulo, se explicara la metodología de trabajo que se utilizo en cada una de las fases del proyecto. 

La realización del sistema, estuvo dividida en dos grandes fases: la primera fase, consistió en hacer investigaciones y estudiar las herramientas a utilizar y la segunda fase estuvo centrada en el diseño, la implementación y la ejecución de las pruebas correspondientes al  sistema. Ambas fases fueron tratadas con metodologías diferentes que serán explicadas a continuación.

\section{Explicación de la Metodología Aplicada}


El proyecto estuvo conformado por varias etapas. En primer lugar,se realizó un estudio sobre los IDS y sus tipos; el sistema SSM, los URI y el protocolo HTTP. Luego, se hizo un análisis de la herramienta Bro, para después diseñar la arquitectura del sistema, realizar la implementación del mismo y hacer las pruebas correspondientes.

Para trabajar las dos primeras fases del proyecto realizado, se estableció un periodo de tiempo para cada una de las misma en una reunión establecida con el tutor industrial.
Entre cada uno de los periodos establecidos en cada fase, existían reuniones con el tutor para solventar las dudas teóricas que iban surgiendo a medida que se realizaban las investigaciones.

El resto de las etapas fue abordada mediante una metodología ágil. Este tipo de métodos,  promueven el desarrollo del software de manera iterativa en donde el ciclo de diseñar, codificar y realizar pruebas se repita de manera continua durante el proceso de desarrollo del software. El uso de este tipo de metodología incrementa el ciclo de vida de los productos\cite{agil}.

La organización de la segunda fase del proyecto consistió ir desarrollando cada uno de los módulos del sistema de manera iterativa, es decir, se diseñaba la arquitectura de uno de los módulos, se implementaban las funcionalidades, se realizaban las pruebas funcionales y luego se iniciaba la otra iteración con el próximo módulo. Al inicio de cada una de las iteraciones, se realizaba una reunión en donde se plantean las tareas que se debían realizar, junto con el tiempo estimado que tomarían las mismas. En dicha reunión se discutían, se ajustaban y si era necesario se agregan tareas a la lista.  Durante el transcurso de las  iteraciones se pautaron reuniones cada dos semanas en donde se mostraba el producto al tutor industrial y se planteaban las dudas que habían surgido durante el desarrollo del proyecto.

Cuando se culminaba de desarrollar la arquitectura, implementar y probar la funcionalidad de un módulo se planificaban las tareas del próxima iteración, y se volvía a realizar un reunión para ajustar las mismas. Este procedimiento se llevó a cabo con los tres módulos que conforman el sistema, es decir, el desarrollo del proyecto, estuvo conformado por tres iteraciones. 
Luego de  finalizar las tres iteraciones para desarrollar los tres módulos que conforman el sistema, se procedió a realizar pruebas operativas sobre el mismo.

El proyecto estuvo conformado por varias etapas. En primer lugar,se realizó un estudio sobre los IDS y sus tipos; el sistema SSM, los URI y el protocolo HTTP. Luego, se hizo un análisis de la herramienta Bro, para después diseñar la arquitectura del sistema, realizar la implementación del mismo y hacer las pruebas correspondientes.

Para trabajar las dos primeras fases del proyecto realizado, se estableció un periodo de tiempo para cada una de las misma en una reunión establecida con el tutor industrial.
Entre cada uno de los periodos establecidos en cada fase, existían reuniones con el tutor para solventar las dudas teóricas que iban surgiendo a medida que se realizaban las investigaciones.

El resto de las etapas fue abordada mediante una metodología ágil. Este tipo de métodos,  promueven el desarrollo del software de manera iterativa en donde el ciclo de diseñar, codificar y realizar pruebas se repita de manera continua durante el proceso de desarrollo del software. El uso de este tipo de metodología incrementa el ciclo de vida de los productos.

La organización de la segunda fase del proyecto consistió ir desarrollando cada uno de los módulos del sistema de manera iterativa, es decir, se diseñaba la arquitectura de uno de los módulos, se implementaban las funcionalidades, se realizaban las pruebas funcionales y luego se iniciaba la otra iteración con el próximo módulo. Al inicio de cada una de las iteraciones, se realizaba una reunión en donde se plantean las tareas que se debían realizar, junto con el tiempo estimado que tomarían las mismas. En dicha reunión se discutían, se ajustaban y si era necesario se agregan tareas a la lista.  Durante el transcurso de las  iteraciones se pautaron reuniones cada dos semanas en donde se mostraba el producto al tutor industrial y se planteaban las dudas que habían surgido durante el desarrollo del proyecto.

Cuando se culminaba de desarrollar la arquitectura, implementar y probar la funcionalidad de un módulo se planificaban las tareas del próxima iteración, y se volvía a realizar un reunión para ajustar las mismas. Este procedimiento se llevó a cabo con los tres módulos que conforman el sistema, es decir, el desarrollo del proyecto, estuvo conformado por tres iteraciones. 
Luego de  finalizar las tres iteraciones para desarrollar los tres módulos que conforman el sistema, se procedió a realizar pruebas operativas sobre el mismo.
