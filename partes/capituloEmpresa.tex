\chapter{ENTORNO EMPRESARIAL}
\label{capituloEmpresa}

A continuación se dará una breve explicación de La Escuela Técnica Superior de Ingenierías Informática de la Universidad de Granada. Lugar en donde se desarrolló el proyecto que se explica en el presente trabajo.

\section{Descripción de la Empresa}
La Universidad de Granada fue fundada en 1531, siendo continuadora de una larga tradición docente que enlaza con la de la Madraza del último Reino Nazarí.

La Universidad está muy presente en la ciudad de Granada, disfrutando de la peculiar belleza de su entorno y de una situación geográfica privilegiada. 

En Granada hay cuatro Campus Universitarios, además del ``Campus Centro'', en el que se integran todos los centros dispersos por casco histórico de la ciudad. Hay otros dos Campus de la UGR en las ciudades de Ceuta y Melilla, en el norte de África.

En la UGR estudian más de 60.000 alumnos de grado y posgrado y otros 10.000 realizan cursos complementarios, de idiomas, de verano, etc. Imparten docencia 3.500 profesores y trabajan más de 2.000 administrativos, técnicos y personal de servicios.\cite{ugr}

Dentro de la UGR se encuentra la ETSII o Escuela Técnica Superior de Ingenierías Informática, que es un centro docente de la Universidad de Granada situado en el campus Aynadamar, junto a la Facultad de Bellas Artes.\cite{etsii}

\section{Misión}
La UGR, creada en 1531, es una universidad
pública, abierta, conectada con su
entorno y con vocación internacional,
comprometida con la innovación, el progreso
y el bienestar social mediante la
mejora continua de la docencia y una
investigación de calidad, la extensión y
difusión de la cultura y la transferencia
del conocimiento.
La UGR se orienta por los valores de
respeto a la dignidad y libertad de las
personas, a la justicia, a la igualdad, a la
solidaridad y a la corresponsabilidad en
el desarrollo sostenible.\cite{ugrMV}
\section{Visión}

La UGR aspira a: \cite{ugrMV}
\begin{itemize}
\item Ser una Universidad bien valorada por las personas y grupos a los que se orienta tanto externos (alumnos potenciales, agentes sociales, organizaciones públicas y privadas) como internos (estudiantes, personal docente e investigador y personal de administración y servicios).
\item Tener un proyecto ético e inteligente que contribuya a un entorno y un mundo mejores, respondiendo y aportando soluciones a las necesidades sociales, culturales, económicas y medioambientales.
\item Distinguirse como una Universidad que aprende, con una formación e investigación de calidad reconocida, dinámica e innovadora.
\item Ser una institución abierta al saber, la innovación,
la crítica, el debate y la sociedad. 
\end{itemize}

\section{Estructura Organizacional}

La Escuela Técnica Superior de Ingenierías Informática está subdividida en cuatro departamentos: Arquitectura y tecnología de computadores, Ciencias de la computación e inteligencia artificial,  Lenguajes y sistemas informáticos y Teoría de la señal, telemática y comunicaciones.\cite{etsiiDepart} 
El departamento en donde se realizó el proyecto fué el departamento de Teoría de la señal, telemática y comunicaciones.


