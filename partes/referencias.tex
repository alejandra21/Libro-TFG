\begin{thebibliography}{99}
\bibitem{pwc} 
PWC: PricewaterhouseCoopers,
\\\texttt{\url{https://www.pwc.com/gx/en/services/advisory/forensics/economic-crime-survey/cybercrime.html}}

\bibitem{salario}
Walters R. (2017), \textit{Salary Survey}, disponible en: \url{https://www.robertwalters.co.uk/content/dam/salary-survey-2017.pdf?webSyncID=6d3bc948-9837-8e7b-91d2-a51713c993b7&sessionGUID=0b156ce4-a2a8-3d94-8cd0-15640ae068ac}

\bibitem{IDS}
H.   Debar (2000), \textit{An Introduction to Intrusion-Detection Systems},  IBM   Research,   Zurich   Research   Laboratory,Ruschlikon, Switzerland.

\bibitem{rfcHTTP}
RFC 7230

\bibitem{rfcURI}
RFC 3986

\bibitem{automata}
Linz P. (2012) \textit{An Introduction to FORMAL LANGUAGE an AUTOMATE}, quinta edición, pp 36.

\bibitem{automata2}
Mishra K.L.P. , Chandrasekaran N. (2008), \textit{THEORY OF COMPUTER SCINCE Automata, Language and Computation},tercera edición, pp 71-72.

\bibitem{automataFinito}
V. Aho A., S. Lam M., Sethi R., D. Ullman J. (2007),\textit{Compilers, Principles, Techniques, \& Tools}, segunda edición,pp 147-150.

\bibitem{markov}
Ching W., Ng M. K. (2006), \textit{Markov Chains: Models, Algorithms and Applications}, pp 1-4.

\bibitem{ejemploMarkov}
Haggstrom (2002) \textit{Finite Markov Chains and Algorithmic Applications}, Lon-
don Mathematical Society, Student Texts 52, Cambridge University Press,Cambridge, U.K.

\bibitem{ssm}
Pedro García-Teodoro.; Jesús E. Díaz-Verdejo; Juan M. Tapiador; Rolando Hernandez-Salazar
Automatic Generation of HTTP Intrusion Signatures by Selective Identification of Anomalies
Computers \& Security, Vol. 55, pp. 159-174, 2015, ISSN: 0167-4048 

\bibitem{tesisMexico}
Rolando Salazar Hernández, Sistema de detección de intrusos mediante modelado de URI. Tesis doctoral. Universidad de Granada. Director: Jesús E. Díaz Verdejo, 02/02/2016

\bibitem{Bro}
\textit{Bro Introduction}, disponible en : \url{https://www.bro.org/sphinx/intro/index.html}

\bibitem{httpKross}
Kurose, J. F., \& Ross, K. W. (2013). \textit{Computer networking: a top-down aproach}. Pearson

\end{thebibliography}