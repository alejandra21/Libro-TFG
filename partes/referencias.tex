\begin{thebibliography}{99}
\bibitem{pwc} 
PWC: PricewaterhouseCoopers,
\\\texttt{\url{https://www.pwc.com/gx/en/services/advisory/forensics/economic-crime-survey/cybercrime.html}}

\bibitem{salario}
Walters R. (2017), \textit{Salary Survey}, disponible en: \url{https://www.robertwalters.co.uk/content/dam/salary-survey-2017.pdf?webSyncID=6d3bc948-9837-8e7b-91d2-a51713c993b7&sessionGUID=0b156ce4-a2a8-3d94-8cd0-15640ae068ac}

\bibitem{IDSdef} Sundar Rajan, S., \& Krishna Cherukuri, V. (2017).\textit{An Overview of Intrusion Detection Systems }. Disponible en: \url{http://www.idt.mdh.se/kurser/ct3340/ht09/ADMINISTRATION/IRCSE09-submissions/ircse09_submission_18.pdf}

\bibitem{IDSimportance} R,K. \& Indra, A.(2010) \textit{Intrusion Detection Tools and Techniques --
A Survey. International Journal Of Computer Theory An Enginering}, 901-906. http://dx.doi.org/10.7763/ijcte.2010.v2.260.

\bibitem{ugr} Granada y la Universidad. Disponible en: \url{https://www.ugr.es/universidad/organizacion/granada-y-la-universidad}

\bibitem{etsii} La Escuela de Informática y Telecomunicación. Disponible en: \url{https://etsiit.ugr.es/pages/escuela}

\bibitem{ugrMV} Misión y Visión de la Universidad de Granada. Disponible en: \url{http://www.ugr.es/~rhuma/sitioarchivos/noticias/MisionVision.pdf}

\bibitem{etsiiDepart} Departamentos que imparten docencia en la ETSIIT. Disponible en: \url{https://etsiit.ugr.es/pages/escuela/departamentos}

\bibitem{IDS}
H.   Debar (2000), \textit{An Introduction to Intrusion-Detection Systems},  IBM   Research,   Zurich   Research   Laboratory,Ruschlikon, Switzerland.

\bibitem{rfcHTTP}
RFC 7230

\bibitem{rfcURI}
RFC 3986

\bibitem{automata}
Linz P. (2012) \textit{An Introduction to FORMAL LANGUAGE an AUTOMATE}, quinta edición, pp 36.

\bibitem{automata2}
Mishra K.L.P. , Chandrasekaran N. (2008), \textit{THEORY OF COMPUTER SCINCE Automata, Language and Computation},tercera edición, pp 71-72.

\bibitem{automataFinito}
V. Aho A., S. Lam M., Sethi R., D. Ullman J. (2007),\textit{Compilers, Principles, Techniques, \& Tools}, segunda edición,pp 147-150.

\bibitem{markov}
Ching W., Ng M. K. (2006), \textit{Markov Chains: Models, Algorithms and Applications}, pp 1-4.

\bibitem{ejemploMarkov}
Haggstrom (2002) \textit{Finite Markov Chains and Algorithmic Applications}, Lon-
don Mathematical Society, Student Texts 52, Cambridge University Press,Cambridge, U.K.

\bibitem{ssm}
Pedro García-Teodoro.; Jesús E. Díaz-Verdejo; Juan M. Tapiador; Rolando Hernandez-Salazar
Automatic Generation of HTTP Intrusion Signatures by Selective Identification of Anomalies
Computers \& Security, Vol. 55, pp. 159-174, 2015, ISSN: 0167-4048 

\bibitem{tesisMexico}
Rolando Salazar Hernández, Sistema de detección de intrusos mediante modelado de URI. Tesis doctoral. Universidad de Granada. Director: Jesús E. Díaz Verdejo, 02/02/2016

\bibitem{GitHub} \textit{About - GitHub}. Disponible en: \url{https://github.com/about}

\bibitem{Git} \textit{GitHub Glossary}. Disponible en: \url{https://help.github.com/articles/github-glossary/}

\bibitem{wireshark1} \textit{Wireshark - about}. Disponible en : \url{https://www.wireshark.org/about.html}

\bibitem{wireshark2} \textit{Wireshark - docs}. Disponible en: \url{https://www.wireshark.org/docs/wsug_html_chunked/ChapterIntroduction.html}

\bibitem{Bro}
\textit{Bro Introduction}, disponible en : \url{https://www.bro.org/sphinx/intro/index.html}

\bibitem{httpKross}
Kurose, J. F., \& Ross, K. W. (2013). \textit{Computer networking: a top-down aproach}. Pearson

\bibitem{Cormen} Cormen H. C., Leiserson E. C., Rivest L. R., Stein C. (2009).Introduction to algorithms, tercera edición, pp 153-155.

\bibitem{wireshark} Capturas de ejemplo de “Wireshark”. Disponible en:  \url{https://wiki.wireshark.org/SampleCaptures?action=AttachFile&
do=view&target=http.cap}

\bibitem{firmas} García-Teodoro P., Díaz-Verdejo J.E., Tapiador J.E., Salazar-Hernández R. (2015) \textit{Automatic generation of HTTP intrusion of anomalies}

\end{thebibliography}