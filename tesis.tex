\documentclass[letterpaper, 12pt, oneside]{tesis}

\usepackage{amsmath}

% Paquetes para idioma
\usepackage[spanish]{babel}
\usepackage[utf8]{inputenc}
\usepackage[fixlanguage]{babelbib}

% Otros paquetes instalados
% Básicos
%\usepackage{natbib}
\usepackage{enumerate}
\usepackage{enumitem}

% Para dibujar figuras
\usepackage{tikz}

% Para cambiar el color de las letras
\usepackage{color}

% Para incluir código (básico)
\usepackage{verbatim}
\usepackage{fancyvrb}

% Para incluir hipervínculos
\usepackage{hyperref}
\hypersetup{urlcolor=blue, colorlinks=false}

% Para más símbolos matemáticos
\usepackage{amsmath}
\usepackage{amsthm}
\usepackage{amssymb}

% Para colocar teoremas en cajas
\usepackage{mdframed}

% Para texto Lorem Ipsum
\usepackage{blindtext}

% Paquetes locales
% Puedes agregar paquetes locales (archivos .sty) en un subdirectorio % 'paquetes'.
% Utiliza la sintaxis \usepackage{paquetes/nombrePaquete}

% Todas las imágenes se cargan del subdirectorio 'img' por defecto.
\graphicspath{{img/}}

% Sangrías de 3 espacios (3 veces el espacio de la x)
\parindent 3ex 

% Interlineado
\setlength{\baselineskip}{1.5pt}

% Interpárrafo
\setlength{\parskip}{16.5pt}

\topmargin 2cm

\renewcommand{\tablename}{Tabla}
\newcommand\listsymbolname{Acrónimos y Símbolos}

\begin{titlepage}
    \title{\vspace{-2cm} \includegraphics[width=1.2in]{./usb.png} \\[.2cm]
        \large Universidad Simón Bolívar \\
        Decanato de Estudios Profesionales \\
        Coordinación de Ingeniería de la Computación
        \vfill \LARGE Módulo para la detección de
ataque mediante Bro \vfill}
    \author{Por: \\
        Cordero García Alejandra \\
        Tutor académico: \\
        Alvárez Kitty \\[1.2cm]
        Tutor industrial: \\
        Díaz Verdejo Jesús \\[1.2cm]
        PASANTÍA LARGA \\
Presentado ante la Ilustre Universidad Simón Bolívar \\
como requisito parcial para optar al título de \\
Ingeniero de Computación}
    \date{Sartenejas, Octubre de 2017}
\end{titlepage}

\begin{document}
\frontmatter
\maketitle
\setstretch{1.3}

% Se incluye el acta de evaluación, verificar que se corresponda
% con el formato aceptado actualmente por el Decanato.
% Pagina del acta final
\begin{titlepage}
\begin{center}

% Upper part
\includegraphics[scale=0.5]{usb.png} \\

\textsc {\large UNIVERSIDAD SIMÓN BOLÍVAR} \\
\textsc{DECANATO DE ESTUDIOS PROFESIONALES\\
COORDINACIÓN DE INGENIERÍA DE LA COMPUTACIÓN}

\bigskip
\bigskip
\bigskip
\bigskip
\bigskip
\bigskip

% Title
\textsc{ACTA FINAL PASANTÍA LARGA}

\bigskip
\bigskip

\textsc{\bfseries MÓDULO PARA LA DETECCIÓN DE
ATAQUE MEDIANTE BRO}

\bigskip
\bigskip
\bigskip
\bigskip

\begin{minipage}{\textwidth}
\centering
Presentado por: \\
\textsc{\bfseries Cordero García Alejandra} \\

\bigskip
\bigskip
\bigskip

Este Proyecto de Grado ha sido aprobado por el siguiente jurado examinador: \\

\bigskip
\bigskip

% Despues de cada line coloca el (los) nombre(s) de
% cada uno de los integrantes del jurado.
\line(1,0){200} \\
Kity Álvarez\\

\bigskip
\bigskip

\line(1,0){200} \\
@jurado1\\

\bigskip
\bigskip

\line(1,0){200} \\
@jurado2\\
\end{minipage}

\bigskip
\bigskip
\vfill

% Date/Fecha
{\large \bfseries Sartenejas, @día de @mes de 2017}

\end{center}
\end{titlepage}
 

% El resumen debe ser de una sola página
\addtotoc{Resumen}
\abstract{
\addtocontents{toc}{\vspace{1em}}

Esta pasantía larga consistió en desarrollar e implementar los
módulos necesarios para implementar un detector de intrusiones para servicios
web, basado en SSM (por el ingles,“Segmented Stochastic Modelling”)[11],
que se pueda utilizar en redes de explotación. El detector fué implementado
haciendo uso del lenguaje de “scripting” de un sistema de monitorización
llamado Bro. Este sistema, se encargará de tomar peticiones HTTP de tipo
GET, extraer sus URIs y analizar si los mismas poseen anomalías de una
manera probabilística, utilizando modelos de normalidad. En el caso en el
que se perciba alguna anormalidad, el sistema se encargará de emitir una
alarma.
El proyecto estuvo dividido en varias fases: La primera fase consistió en
estudiar SSM y sus conceptos relacionados, así como familiarizarse con la
herramienta Bro; la segunda fase se basó en diseñar y dividir las tareas del
sistema en tres módulos principales: un modulo de segmentación, un módulo
de evaluación y un modulo de entrenamiento; en la tercera fase se realizó la
implementación y las pruebas funcionales del detector de intrusiones; en la
cuarta fase se realizaron las pruebas operativas al sistema completo haciendo
uso de bases de datos con trazas normales y bases de datos con trazas
anómalas; y finalmente,en la sexta fase, se estructuró y se redactó la memoria
del proyecto.


% Las palabras clave son generalmente los nombres de áreas de investigación a
% los cuales está asociado el trabajo. Generalmente son tres o cuatro.
\noindent \begin{small} \textbf{Palabras clave}: SSM, HTTP, GET, URI, Bro.
\end{small}

% Iniciar nueva página luego del resumen
\clearpage
\setstretch{1.3}

% Agradecimientos
\acknowledgements{
\addtocontents{toc}{\vspace{1em}}
Agradezco en primer lugar a mis padres por darme su apoyo incondicional
siempre. También debo darle las gracias a mis amigos por brindarme su
ayuda y al Dr. Díaz Verdejo por su paciencia y dedicación.
}
\clearpage

\pagestyle{fancy}

% Tabla de contenidos o índice
\lhead{\emph{Índice General}}
\tableofcontents

% Estos índices solamente se usan si el libro contiene figuras, tablas y
% algoritmos. Si alguno de estos no se utiliza, comentar o eliminar las líneas
% pertinentes.
\lhead{\emph{Índice de Figuras}}
\listoffigures

\lhead{\emph{Índice de Tablas}}
\renewcommand*\listtablename{Índice de Tablas}
%\listoftables

%\lhead{\emph{Índice de Algoritmos}}
%\renewcommand*\listalgorithmname{Índice de algoritmos}
%\listofalgorithms

\setstretch{1.5}
\clearpage
\lhead{\emph{Acrónimos y símbolos}}
\listofsymbols{ll}
{

    % Aquí van las siglas
    \textbf{SIGLAS} & \textbf{S}iglas \textbf{I}sla \textbf{G}rafo 
                      \textbf{L}aos \textbf{A}ve \textbf{S}erpiente\\
    \textbf{ACM} & \textbf{A}ssociation for \textbf{C}omputing \textbf{M}achinery\\
    &\\
    \hline
    &\\

    % Aquí van los símbolos
    $\iff$ & doble implicación, si y sólo si\\
    $\Rightarrow$ & implicación lógica\\
    $[u:=v]$ & sustitución textual de $u$ por $v$
}

%% ----------------------------------------------------------------
% End of the pre-able, contents and lists of things
% Begin the Dedication page

\setstretch{1.3}  % Return the line spacing back to 1.3

\pagestyle{empty}  % Page style needs to be empty for this page

\dedicatory{
    \textbf{Dedicatoria} \bigskip

    A @personasImportantes, por @razonesDedicatoria.
}

\addtocontents{toc}{\vspace{2em}}

\mainmatter
\pagestyle{fancy}

% Se incluye el cuerpo de la tesis en este documento.

\chapter*{INTRODUCCIÓN}
\label{intro}
\lhead{\emph{Introducción}}
\addcontentsline{toc}{chapter}{Introducción}

% Descripción del problema, de lo general hacia lo específico

Hoy en dia es muy difícil imaginar el mundo sin el Internet. Esta gran red de
redes se ha convertido en una parte importante en la comunicación en el mundo.
Un gran porcentaje del uso que se le da al Internet proviene de las aplicaciones
web. Para compartir toda la información brindada por estas aplicaciones a través del Internet, se hace uso del protocolo HTTP ("Hypertext Transfer Protocol"), un protocolo cliente-servidor. Los servidores HTTP, son los encargados de suplir las peticiones de sus clientes y enviar o recibir la información por los mismos.
Este tipo de servidores son un punto apetecible para los hackers ya que los
mismos contienen informacion importante de las aplicaciones web. Como acceso a
la base de datos de la aplicación que contiene las claves, los nombres de usuarios
y número de tarjetas de crédito. Por esta razón, es de total importancia velar por su seguridad.

Este punto será el tema central en el presente trabajo. En el
mismo, explicará la implementación y las bases téoricas de un protocolo desarrollado en
la Universidad de Granada llamado SSM ("Segmented Stochastic Modelling") en un
lenguaje se "scripting" del analizador de redes Bro.

El objetivo de implementar este sistema es el de crear un detector de instrusiones
(IDS) que detecte de manera adecuada intrusiones en servidores de tipo
HTTP.

La estructura del siguiente trabajo será la siguiente:
En el capítulo uno se hablará sobre el planteamiento del problema,
los objetivos tanto generales como especificos y la planificación que se utilizó para
realizar el proyecto.
Por otra parte,en el segundo capítulo se tocarán temas relacionados con el
estado del arte del proyecto. En primero lugar se explicará la definición de IDS y
los tipos de IDS que existen, se explicará la composición de los URI según el RFC
(mencionar el numero del RFC) y se dará una descripción del funcionamiento SSM
los modelos que lo conforman. Además, se describirá la herramiento utilizada
y el lenguaje de "scripting" que se utilizó para implementar el sistema.
En el tercer capítulo se hablará sobre el diseño del sistema. Aquí se describirán
los módulos en los que esta dividido el IDS y como es el funcionamiento de cada
uno. Los detalles de implementacion de estos se explicarán en el capítulo cuarto.
El quinto capítulo tocará el tema de la evaluación y las pruebas. En este capítulo
se hablará sobre la base de datos utilizada para realizar las pruebas y los
resultados arrojados por las mismas. Además se describirán un poco las pruebas
tanto operativas como funcionales realizadas sobre el sistema.

% Trabajos anteriores

% Objetivo general

% Objetivos específicos

% Organización del trabajo

% Se describe brevemente qué se hace en cada capítulo




% El número de capítulos varía. En mi libro fueron cuatro (sin contar
% introducción y conclusión).
\chapter{Introducción}
\label{capitulo1}
\lhead{Capítulo 1.}

% De qué va a tratar el capítulo
% El capítulo 1 suele ser el marco teórico.

\section{Objetivos}
\subsection{Objetivos generales}
El objetivo del presente trabajo es incorporar la técnica de detección de intrusiones SSM propuesta en [1] en el sistema Bro. Para ello deben usarse las capacidades de scripting de Bro e incorporar las funcionalidades necesarias como un módulo que pueda activarse a demanda para monitorizar las peticiones a los servicios web.
\subsection{Objetivos especificos}
La consecución del objetivo general planteado puede abordarse a partir de los siguientes objetivos específicos:

\begin{enumerate}
\item Interceptar las URI para su análisis haciendo uso de las herramientas presentadas por Bro para capturar y filtrar la información de los paquetes del protocolo HTTP.
\item Desarrollar un módulo que permita evaluar el índice de anomalía de los URI.
Se diseñará e implementará un módulo de evaluación haciendo uso de las herramientas otorgadas por el lenguaje de scripting de Bro y siguiendo las especificaciones y expresiones pautadas por la técnica de detección de intrusiones SSM. 
\item Desarrollar un módulo para la estimación de los modelos de normalidad
Se diseñará e implementará un módulo de entrenamiento haciendo uso de las herramientas otorgadas por el lenguaje de scripting de Bro que permita obtener un modelo de normalidad a partir de trafico libre de ataques.
\item Evaluar experimentalmente el funcionamiento del sistema
Se realizarán pruebas, tanto funcionales como operativas para verificar el correcto funcionamiento del sistema.
\end{enumerate}

\section{Planificación}

La tareas en las que se dividió el desarrollo de este proyecto fueron las siguientes:

\newlist{legal}{enumerate}{10}
\setlist[legal]{label*=\arabic*.}

\begin{legal}
\item Estado del arte
Durante esta etapa se realizó un estudio sobre los IDS, los tipos de IDS y el sistema SSM. Así mismo, se hizo una lectura del RFC 3986 (URI) y se investigó sobre el protocolo HTTP.

\item Análisis de Bro 
Durante esta etapa del proyecto se instaló, se estudio la documentación y se aprendió a hacer uso de la herramienta Bro y su lenguaje de scripting.

\item Arquitectura modular del sistema 

En esta etapa se realizó el diseño de los módulos principales que conforman el sistema. Esta tarea se basó en dividir el trabajo de cada uno de los módulos en varias funcionalidades y en diseñar las estructuras de datos y el formato de los archivos de texto del sistema.

    Esta tarea estuvo subdividida en las siguientes subtareas:

\begin{legal}
\item Diseño del módulo de análisis sintáctico/segmentación de URIS 
\item Diseño del módulo de evaluación de URIs 
\item Diseño del módulo de entrenamiento 
\end{legal}
\item Implementación del sistema 
En esta etapa del proyecto se llevó a cabo la implementación del sistema haciendo uso del lenguaje de scripting de Bro.

\begin{legal}
\item Módulo de análisis sintáctico/segmentación de URIS 

En esta etapa se tomaron los diseños esbozados previamente y se implementaron todas las funcionalidades necesarias para realizar el segmentador y el analizador sintáctico de los URIS. Fue necesario hacer uso de expresiones regulares y de las herramientas otorgadas por Bro para manejarlas.

\item Módulo de evaluación de URIs 

En esta etapa se tomaron los diseños realizados y se implementaron todas las expresiones, las estructuras de datos y las funcionalidades necesarias para construir un modulo de evaluacion funcional haciendo uso de las herramientas presentadas por Bro. 

\item Módulo de entrenamiento 

En esta etapa se implementaron las estructuras de datos y los dos modos de entrenamiento (Online y Offline) haciendo uso del lenguaje de scripting de Bro.

\item Módulo de gestión del tipo de modo del sistema
En esta etapa del proyecto se implementó el módulo de gestión del tipo de modo del sistema haciendo uso de bash scripting. Mediante este módulo se escogerá en que tipo de modo correrá el sistema (entrenamiento o evaluación).

\item Módulo de reportes
En esta etapa se implementaron los reportes del sistema haciendo uso de herramientas de Bro dedicadas exclusivamente a realizar este tipo de tareas.

\end{legal}
\item Evaluación y pruebas 
Una vez finalizada la implementación del sistema se procedió a realizar pruebas tanto funcionales como operativas.
\begin{legal}
\item Realización de pruebas funcionales

En esta etapa del proyecto se hicieron pruebas de manera individual a cada una de las funciones más importantes de los módulos que conforman el sistema y se realizaron las modificaciones adecuadas en las mismas.

\item Realización de pruebas operativas

La realización de las pruebas operativas se basó en tomar base de datos de trazas tanto normales como anormales, probar el funcionamiento de todo el sistema y realizar las modificaciones adecuadas.
\end{legal}

\item Documentación  
 Esta fase consistió en estructurar y redactar el libro del proyecto.
\end{legal}

A continuación, se mostrará un diagrama de Gantt en donde se muestra el tiempo que tomó cada una de las tareas.

\begin{center}
\includegraphics[width=4in]{Proyecto-grado.png}
\end{center}


\section{Presupuesto}

A continuación se mostrará una tabla que muestra el presupuesto del proyecto realizado.

\begin{center}
\includegraphics[width=4in]{presupuesto.png}
\end{center}
\chapter{Entorno Empresarial}
\label{capituloEmpresa}

A continuación se dará una breve explicación de La Escuela Técnica Superior de Ingenierías Informática de la Universidad de Granada. Lugar en donde se desarrolló el proyecto que se explica en el presente trabajo.

\section{Descripción de la Empresa}
La Universidad de Granada fue fundada en 1531, siendo continuadora de una larga tradición docente que enlaza con la de la Madraza del último Reino Nazarí.

La Universidad está muy presente en la ciudad de Granada, disfrutando de la peculiar belleza de su entorno y de una situación geográfica privilegiada. 

En Granada hay cuatro Campus Universitarios, además del ``Campus Centro'', en el que se integran todos los centros dispersos por casco histórico de la ciudad. Hay otros dos Campus de la UGR en las ciudades de Ceuta y Melilla, en el norte de África.

En la UGR estudian más de 60.000 alumnos de grado y posgrado y otros 10.000 realizan cursos complementarios, de idiomas, de verano, etc. Imparten docencia 3.500 profesores y trabajan más de 2.000 administrativos, técnicos y personal de servicios.\cite{ugr}

Dentro de la UGR se encuentra la ETSII o Escuela Técnica Superior de Ingenierías Informática, que es un centro docente de la Universidad de Granada situado en el campus Aynadamar, junto a la Facultad de Bellas Artes.\cite{etsii}

\section{Misión}
La UGR, creada en 1531, es una universidad
pública, abierta, conectada con su
entorno y con vocación internacional,
comprometida con la innovación, el progreso
y el bienestar social mediante la
mejora continua de la docencia y una
investigación de calidad, la extensión y
difusión de la cultura y la transferencia
del conocimiento.
La UGR se orienta por los valores de
respeto a la dignidad y libertad de las
personas, a la justicia, a la igualdad, a la
solidaridad y a la corresponsabilidad en
el desarrollo sostenible.\cite{ugrMV}
\section{Visión}

La UGR aspira a: \cite{ugrMV}
\begin{itemize}
\item Ser una Universidad bien valorada por las personas y grupos a los que se orienta tanto externos (alumnos potenciales, agentes sociales, organizaciones públicas y privadas) como internos (estudiantes, personal docente e investigador y personal de administración y servicios).
\item Tener un proyecto ético e inteligente que contribuya a un entorno y un mundo mejores, respondiendo y aportando soluciones a las necesidades sociales, culturales, económicas y medioambientales.
\item Distinguirse como una Universidad que aprende, con una formación e investigación de calidad reconocida, dinámica e innovadora.
\end{itemize}

\section{Estructura Organizacional}

La Escuela Técnica Superior de Ingenierías Informática está subdividida en cuatro departamentos: Arquitectura y tecnología de computadores, Ciencias de la computación e inteligencia artificial,  Lenguajes y sistemas informáticos y Teoría de la señal, telemática y comunicaciones.\cite{etsiiDepart} 
El departamento en donde se realizó el proyecto fué el departamento de Teoría de la señal, telemática y comunicaciones.



\chapter{Marco teórico}
\label{capitulo2}
\lhead{Capítulo 2. \emph{Marco teórico}}


En el cap\'itulo se describir\'an algunos conceptos importantes para comprender la implementaci\'on del sistema  en el que se basa el presente trabajo. Dichos conceptos son: \textit{IDS, el protocolo HTTP, URI , autom\'atas de estados finitos, autom\'atas probabilísticos, modelo de Markov,SSM y Bro}.

En l\'ineas generales, se tiene como prop\'osito explicar el funcionamiento de un \textit{IDS} basado en anomalías que haga uso del sistema \textit{SSM} y del lenguaje de scripting de la herramienta \textit{BRO}.

Este sistema analiza los \textit{URI} de las peticiones tipo GET o HEAD del protocolo \textit{HTTP}, haciendo uso de la t\'ecnica \textit{SSM}. Esta t\'ecnica, a su vez hace uso de \textit{autómatas de estados finitos probabilísticos} y del \textit{modelo de Markov} para determinar si las peticiones realizadas a un servidor \textit{HTTP} son posibles ataques o no.
% De qué va a tratar el capítulo
% El capítulo 1 suele ser el marco teórico.

\section{Protocolo HTTP}

El protocolo HTTP, (``Hypertext Transfer Protocol'') es un protocolo de la capa de aplicación que se encuentra en el corazón de la Web. Se define en [RFC 1945] y [RFC 2616]. HTTP esta implementado en dos programas: un programa cliente y un programa servidor. El programa cliente y el programa servidor se ejecutan en diferentes sistemas y se comunican entre si intercambiando mensajes HTTP. HTTP por su parte, define la estructura de dichos mensajes y como se realiza el intercambio de los mismos. \cite{httpKross}

El protocolo HTTP, es un protocolo simple de petición-respuesta que funciona normalmente sobre la capa de transporte TCP. La idea general de la interacción entre el cliente y el servidor se muestra en la figura \ref{fig:httpPeticionRespuesta}. 

\begin{figure}[tb]
\begin{center}
\includegraphics[width=3in]{HTTP.png}
\caption{Petición-respuesta.\cite{httpKross}}
\label{fig:httpPeticionRespuesta}
\end{center}
\end{figure}

En el protocolo HTTP, el cliente inicia la conexión TCP con el servidor. Una vez establecida la conexión, ambos podrán enviar y recibir mensajes de petición y respuesta. La información que se transfiere mediante este protocolo pueden ser archivos de texto plano, hipertexto, audio, imágenes o cualquier información accesible por Internet.

Es importante tener en cuenta que el servidor envía los archivos solicitados a los clientes sin almacenar información del estado de los mismo, es decir, el servidor no almacena información de sus clientes. Debido a esto se dice que HTTP es un protocolo sin estado.\cite{httpKross}

Por otra parte, los mensajes de petición y respuesta del protocolo HTTP poseen una estructura general como la que se muestra en la figura \ref{fig:httpMensaje}.

\begin{figure}[tb]
\begin{center}
\includegraphics[width=3in]{mensajeHTTP.png}
\caption{Mensaje HTTP.\cite{httpKross}}
\label{fig:httpMensaje}
\end{center}
\end{figure}

La linea de petición en los mensajes HTTP, presenta tres partes separadas por espacios:

\begin{enumerate}
\item El método.
\item El identificador del recurso solicitado (URI).
\item La version del protocolo HTTP.
\end{enumerate}

El método es un parámetro que indica lo que actualmente se esta solicitando por el mensaje. 
Los m\'etodos GET, HEAD y POST son los más utilizados en las comunicaciones del protocolo HTTP. 

\begin{itemize}
\item GET se encarga de hacer solicitudes de recursos al programa servidor HTTP.
\item HEAD solicita recursos de la misma forma en la que lo hace el método GET con la \'unica diferencia de que la solicitud solo ser\'a respondida con la cabecera del recurso, excluyendo el cuerpo del mismo.
\item POST solicita que el recurso de destino  procese la informaci\'on que viene incluida en la solicitud realizada
\end{itemize}

Para el presente trabajo la sintaxis de los URI y los mensajes de petición de tipo GET resultan de importante relevancia.

\section{URI} \label{URIsection}

Un URI es una serie de caracteres que identifican un recurso en la red. Este posee una sintaxis específica que está conformada por diferentes segmentos de caracteres: el esquema, la autoridad, la ruta, el ``query'' y el ``fragment''.

En la figura \ref{fig:URI}, se muestra, en un ejemplo de los componentes que forman un URI.

\begin{figure}[H]
\begin{center}
\includegraphics[width=3in]{./URI.jpg}
\caption{URI.}
\label{fig:URI}
\end{center}
\end{figure}

A continuaci\'on se describe de manera breve cada uno de los componentes que pueden conformar un URI.

\begin{itemize}

\item Esquema:

El esquema identifica el protocolo. En el caso del presente trabajo, los URIs utilizaran únicamente el protocolo http o https.


\item Autoridad:

Este componente del URI posee información del ``host'' del URI, el cual viene representado bien sea por un IPv4 o un nombre de dominio, seguido de un número de puerto (opcional).
   
\item ``Path'':
La ruta de un URI es un conjunto de segmentos organizados de manera jerárquica que contienen información sobre la ubicación de los recursos a solicitar.

\item ``Query'':

El ``query'' de un URI es un segmento de información no jerarquizada, cuyo símbolo de inicio es el signo de interrogación (?). Por lo general, el ``query'' está conformado por una dupla ``atributo=valor'' que junto con la ruta identifican el recurso que se desea solicitar. A diferencia de la ruta, la información aquí contenida debe ser procesada por el servidor al que se le está solicitando el recurso. 

\item ``Fragment'':

En un URI, el ``fragment'' corresponde a la dirección de un segundo recurso dentro del primer recurso identificado por la ruta y el ``query''. Esta cadena de caracteres esta precedida por el símbolo de un numeral (\#).

\end{itemize}

En un URI, tanto el esquema como la ruta son segmentos que deben existir de manera obligatoria. Si la ruta es un carácter vac\'io, se asumir\'a que la misma es ``/''. El resto de los segmentos son opcionales.

\section{Detector de intrusiones (IDS)}
Un detector de intrusiones (IDS del inglés "Intrusion-detection system" ) es un sistema que se encarga de procesar la información entrante de un sistema a proteger con la intención de detectar actividades maliciosas y lanzar alertas para activar el proceso de auditoría por parte de los administradores de sistema \cite{IDS}

Los IDS suelen clasificados en: IDS basados en firmas o 		IDS 	basados en anomalías. 

Los IDS basados en firmas, utilizan una base de datos de ataques conocidos y vulnerabilidades del sistema los cuales son utilizados para realizar comparaciones con los patrones seguidos por las actividades detectadas. Si se detecta similitud con alguno de los patrones de la base de datos, se activará una alarma para indicar que un ataque esta siendo perpetuado.

La eficacia de los detectores de intrusiones basados en firmas es muy buena, sin embargo, su buen funcionamiento depende completamente de la constante actualización de la base de datos de ataques.

Por otra parte, los IDS basados en anomalía detectan los ataques de manera diferente. En este tipo de IDS los ataques son detectados a partir de la observación de comportamientos, bien sea del sistema, o de los usuarios. 

El modo de funcionamiento de un IDS basado en anomalía se basa en: recolectar la información de las actividades detectadas y comparar dicha información con un modelo de normalidad del sistema que ha sido previamente construido. El modelo de normalidad se construye a partir de comportamientos previamente observado en el sistema y que son catalogados como "normales". Si existe una incongruencia entre la información entrante y el modelo de normalidad, entonces se generará una alarma.

No obstante, uno de los problemas de este tipo de detector de intrusiones es la alta tasa de falsos positivos que puede llegar a tener. Esto se debe, a que los modelos, por lo general, no contienen todos los comportamientos normales de un sistema en su totalidad. Esto, a su vez provoca que existan eventos libres de ataques que sean catalogados como una amenaza. Tambi\'en está el hecho de que es muy posible que con el pasar del tiempo, el comportamiento del sistema a proteger vaya cambiando, lo cual implicar\'ia que, si no se hace una constante revis\'on y reentrenamiento del modelo de normalidad el mismo quedar\'a obsoleto y los eventos entrante ser\'an mal catalogados por el IDS. El problema de realizar entrenamientos constantes para actualizar un modelo de normalidad es que se crearán brechas de tiempo en donde el sistema ser\'a muy vulnerable a ataques ya que en lugar de estar funcionando el módulo para detección, estar\'a funcionando el módulo para entrenar el modelo de normalidad. Por lo tanto, si un ataque es perpetrado durante ese periodo de tiempo, el mismo quedar\'a grabado en el modelo de normalidad como un comportamiento normal del sistema.

También existen los IDS híbridos cuyo funcionamiento mezcla el funcionamiento de los detectores de intrusiones basados en firmas y los basados en anomalía, es decir, este tipo de  IDS suelen contar con una base de datos de firmas y también con un modelo de normalidad.

\section{Autómatas de estados finitos}

Un autómata es un modelo abstracto de un ordenador digital \cite{automata}. Una manera de definirlo de forma gráfica es como se muestra en la figura \ref{fig:automata}.

\begin{figure}[H]
\begin{center}
\includegraphics[width=3in]{automata.png}
\caption{Autómata.\cite{automata2}}
\label{fig:automata}
\end{center}
\end{figure}

donde \cite{automata2}:

\begin{itemize}
\item  $I_{1}, I_{2},..., I_{n}$ es el "input" del autómata  que es introducido en cada instante de tiempo $t_{1}, t_{2},..., t_{n}$.

\item $O_{1}, O_{2},..., O_{n}$ es la salida del autómata.

\item $q_{1}, q_{2},..., q_{n}$ son los estado en los que puede estar el autómata en cualquier instante de tiempo.
\end{itemize}

Dentro de los autómatas existe la familia de los autómatas de estado finito. Un autómata de estado de autómata finito se puede definir formalmente como una quíntupla \cite{automataFinito}:.

\begin{equation}
M = (Q,\Sigma,\delta,q,F)
\end{equation}

donde:

\begin{itemize}

\item Q es un conjunto finito de estados.

\item  $\Sigma$ es un conjunto finito, llamado alfabeto. Los elementos de $\Sigma$ se llaman símbolos.

\item $\delta$ es una función de transición, tal que $ Q x \Sigma \leftarrow Q$
 
\item  q  es un elemento perteneciente a Q llamado estado inicial.

\item F es un subconjunto de Q y representa el conjunto de estados finales.

\end{itemize}

Los autómatas finitos se clasifican en autómatas finitos deterministas (DFA, del inglés "Deterministic Finite Automata") y no-deterministas (NFA, del inglés "Nondeterministic Finite Automata").

Los autómatas finitos deterministas se definen teóricamente de la misma manera en la que se definen los autómatas finitos. Sin embargo, se debe cumplir que por cada estado $"s"$ y símbolo de entrada $"a"$ debe existir un único estado al que se puede hacer una transición. Esto quiere decir que la función de transición $\delta$ de los autómatas finitos deben retornar un solo elemento.

Por otra parte, los autómatas finitos no-deterministas al igual que los autómatas finitos deterministas se definen teóricamente de la misma forma que los autómatas finitos. No obstante, la función de transición $\delta$ de estos autómatas puede retornar un conjunto de elementos, es decir, por cada estado $"s"$ y símbolo de entrada $"a"$ pueden existir varios estados al los que se puede hacer una transición.

Tanto los autómatas finitos deterministas como los no deterministas se puede representar a tráves de un grafo. En donde los estados serian representados a tráves de los nodos y las aristas representarían las funciones de transición. 

Un ejemplo de un NFA representado a tráves de un grafo se presenta en la figura \ref{fig:NFA}.

\begin{figure}[H]
\begin{center}
\includegraphics[width=3in]{NFA.png}
\caption{Autómata no-determinista.\cite{automataFinito}}
\label{fig:NFA}
\end{center}
\end{figure}

En la figura \ref{fig:DFA} se muestra la representación de un DFA haciendo uso de un grafo.

\begin{figure}[H]
\begin{center}
\includegraphics[width=3in]{DFA.png}
\caption{Autómata determinista.\cite{automataFinito}}
\label{fig:DFA}
\end{center}
\end{figure}

Los autómatas deterministas permitirían determinar, dado un autómata asociado a un protocolo, si una secuencia de mensajes intercambiados puede ser decodificada mediante dicho autómata. Esto es, si la secuencia de mensajes corresponde a la especificación del protocolo y, por lo tanto, es gramaticalmente correcta. No obstante, una parte importante de los ataques son estructuralmente correctos de acuerdo al protocolo, es decir, siguen la gramática \cite{tesisMexico}.

Sin embargo, los autómatas no deterministas permitirán discriminar ataques cuya estructura sintáctica sea correcta porque permiten incorporar información relacionada con la semántica o el contexto .

Un tipo de autómata no determinista de especial interés para los objetivos del presente trabajo son los autómatas de estado finitos probabilístico [Brookshear, 1989]. En estos se consideran probabilidades asociadas a los estados y/o a los símbolos de forma que se atribuye una naturaleza probabilística tanto a la secuencia de estados que sigue el sistema como a la de los símbolos observados.


\section{Modelo de Markov}

En un proceso de Markov un evento en un tiempo determinado, depende de los procesos inmediatos anteriores a este \cite{markov}. Un modelo de Markov discreto, por otra parte, consiste en un conjunto de estados y una serie de probabilidad de transición que rigen las transiciones entre estados y la producción de símbolos.

Un modelo de Markov discreto, $\lambda$, se define como un quíntupla:

\begin{equation}
\lambda = (Q,\Theta,A,B,\Pi)
\end{equation}

donde:
 
\begin{itemize}
\item Q :  Es el conjunto de N estados del modelo.
\item $\Theta$ : Es el vocabulario del modelo, es decir, son los posibles s\'imbolos o eventos observables del sistema.
\item A : Es una matriz NxN de probabilidades de transici\'on entre estados.
\begin{equation}
\begin{aligned}
A = [a_{ij}], 1\leq i \leq N  y  1\leq j \leq N.\\
a_{ij} = P(q_{t} = S_{j} | q_{t-1} = S_{i})
\end{aligned}
\end{equation}

\item B : Es una matriz NxM de probabilidades de generaci\'on u observación de los s\'imbolos .
\begin{equation}
\begin{aligned}
B = [b_{ik}], 1\leq i \leq N, 1\leq k \leq M.\\
b_{ik} = P(O_{t} = v_{k} | q_{t} = S_{i})
\end{aligned}
\end{equation}

\item $\Pi$ es el vector de probabilidades del estado inicial.
\begin{equation}
\begin{aligned}
\Pi = [\Pi_{i}], 1\leq i \leq N \\
\pi_{i} = P(q_{1} = S_{i})
\end{aligned}
\end{equation}

\end{itemize}

Un ejemplo sencillo de un modelo de Markov sería el siguiente \cite{ejemploMarkov}

S = {$s_{1}$,$s_{2}$}, donde, $s_{1} = "lluvioso"$ y $s_{2} = "soleado"$ y la matriz de transición es:

\[
 P = \begin{pmatrix}
  0.75 & 0.25  \\
  0.25 & 0.75  \\
 \end{pmatrix}
\]

Un modelo de Markov es un autómata de estados finitos no determinista que puede ser representado mediante grafos.

El ejemplo presentado con anterioridad, por lo tanto se puede graficar de la manera en la que se muestra en la figura \ref{fig:modeloMarkov}.

\begin{figure}[H]
\begin{center}
\includegraphics[width=3in]{ejemploMarkov.jpeg}
\caption{Ejemplo modelo Markov.}
\label{fig:modeloMarkov}
\end{center}
\end{figure}


\section{SSM}\label{sec:modeloSSM}

SSM o \textit{Structural Stochastic Modeling} en ingl\'es es una tecnica desarrollado por Estevez-Tapiador. A grandes rasgos, esta t\'ecnica se basa en la definición de un autómata de estados finitos estocástico capaz de evaluar la probabilidad de generación de una petición concreta. El autómata permitirá, por tanto, dada una petición, evaluar si dicha petición es legítima (corresponde al modelo) y su probabilidad. En función de la probabilidad y de un umbral se clasificaran las peticiones como normales o anormales.\cite{ssm}

Esta t\'ecnica, se basa en la teor\'ia de los modelos de Markov ya que se define un autómata de estados finitos que permite evaluar la probabilidad de generaci\'on de una petici\'on, dado un modelo previamente construido.

Por otra parte, en el presente trabajo, las peticiones que ser\'as estudiadas por dicha t\'ecnica ser\'an peticiones de tipo GET del protocolo HTTP. En concreto, de las peticiones GET el elemento a estudiar ser\'an los URIs de dichas peticiones. No obstante, se puede extender para que funcione con m\'etodos de tipo POST y HEAD. 

\begin{figure}
\begin{center}
  \includegraphics[width=\linewidth]{ssm.jpeg}
  \caption{Automata del modelo SSM}
  \label{fig:ssm}
\end{center}

\end{figure}

Este hecho hace que el autómata de SSM deba tomar una cierta topolog\'ia para que de esta forma se puedan reconocer los URIs de cada petici\'on. Dicha topolog\'ia se infiere a tráves de la sint\'axis de los URIs. En la figura \ref{fig:ssm} se muestra la topolog\'ia del autómata.

Las transiciones del autómata vienen dada, por las especificaciones de las sintaxis de los URIs, que est\'an descritas en el RFC 3986. Adem\'as, se puede observar en la figura \ref{fig:ssm} que el autómata tiene un estado inicial ($S_{I}$), un estado del ``host'' ($S_{S}$), un estado de segmento de ruta ($S_{P}$), un estado atributo ($S_{A}$), un estado valor ($S_{V}$), un estado final ($S_{F}$) y un estado sumidero ($S_{OOS}$) en el que se termina si el URI no es sint\'acticamente correcto.

Si una petici\'on del protocolo HTTP anexa un URI que no puede ser reconocido por el autómata descrito con anterioridad implicar\'a que el URI de dicha petici\'on no est\'a bien construido sintácticamente.

Adem\'as, existirá un vocabulario diferente para los estados $S_{S}$, $S_{P}$, $S_{A}$ y $S_{V}$. Estos vocabularios, son construidos, a partir de tomar m\'ultiples peticiones libres de ataques realizadas al servidor. 

Entonces, en resumen, lo que hace la t\'ecnica de SSM para detectar intrusiones es verificar la sintaxis del URI de las peticiones enviadas al servidor a tráves del autómata descrito con anterioridad y, al mismo tiempo, se estudia segmento por segmento del URI la probabilidad que tiene cada palabra de aparecer en un estado determinado del autómata, para as\'i verificar si el URI que viene anexado al la petici\'on es una cadena de caracteres que probabilísticamente corresponde a un petici\'on normal o no.

Teóricamente, el modelo utilizado por SSM es el siguiente \cite{tesisMexico}:

\begin{itemize}
\item Q: es el conjunto de estados ($S_{I}$,$S_{S}$,$S_{P}$,$S_{A}$,$S_{V}$,$S_{F}$,$S_{OOS}$).
\item $\theta$: es el conjunto de s\'imbolos observables que se encuentran en el vocabulario de cada estado.
\item A: es la matriz de probabilidad de transiciones entre estados
\item B: es un conjunto de vectores que contiene la probabilidad de las palabras observadas en cada estado. ($B_{I}$,$B_{S}$,$B_{P}$,$B_{A}$,$B_{V}$,$B_{F}$,$B_{OOS}$).
\item $\Pi$: es el vector de probabilidades iniciales, cuyos valores est\'an determinados por la topolog\'ia del modelo. 
\end{itemize}

\subsection*{Evaluación}
\label{subsec:exprIndice}

Dado un conjuntos de observaciones $0 = o1,o2,...,oT$. Cada una pertenecientes a los estados $Q = q1,q2,..,qT$. En principio, se puede realizar la evaluación de un URI, dado un modelo $\lambda$,  calculando la probabilidad del conjunto completo de observaciones O del modelo $\lambda$ siguiendo el patrón de la secuencia de estados Q. Es decir:

\begin{equation}
P(O|\lambda,Q) = \pi_{q1}b_{q1o1}\prod_{t=1}^{T-1}a_{q_{t}q_{t+1}}b_{q_{t+1}o_{t+1}}
\end{equation}

Para evitar problemas de desbordamiento de calcula el logaritmo de la probabilidad en lugar de la probabilidad.

\begin{equation}\label{eq:Probabilidad}
log(P(O|\lambda,Q)) = log(\pi_{q1}b_{q1o1}\prod_{t=1}^{T-1}a_{q_{t}q_{t+1}}b_{q_{t+1}o_{t+1}})
\end{equation}

Si se considera que las probabilidades iniciales son cero excepto para S1 y que las probabilidades de transición son equivalentes a 1. Entonces, se tendría que la fórmula presentada en \ref{eq:Probabilidad} se podría resumir en la siguiente:

\begin{equation}
P(O|\lambda,Q) = \sum_{t=1}^{T}b_{q_{t}o_{t}}
\end{equation}

Entonces, en primera instancia se podría decir que el índice de anormalidad de un URI U, dado un modelo $\lambda$, se puede obtener de la siguiente forma:

\begin{equation}\label{eq:Probabilidad2}
N_{s} = -log(P(U|\lambda))
\end{equation}

La fórmula \ref{eq:Probabilidad2} será aplicada sólo si el URI U llega al estado final Sf, de otra manera al índice de anormalidad Ns se le será asignado $\infty$, es decir:

\begin{equation}\label{eq:Ns1}
N_{s} = 
	\begin{cases} 
      -log(P(U|\lambda)) & $si $  q_{t} = S_{f} \\
      \infty & $si $  q_{t} \neq S_{f} \\ 
   \end{cases}
\end{equation}

Como se puede observar en la fórmula \ref{eq:Ns1}, mientras menor sea la probabilidad de aparición de las observaciones, mayor será el índice de anormalidad, por esta razón, una vez calculado el índice, se podría decir que un URI U es anómalo, si el índice de anormalidad es mayor o igual a un umbral de detección $\theta$ de lo contrario se dirá que el URI no posee anomalías, esto es:

\begin{equation}\label{eq:ClaseU}
Clase(U) = 
	\begin{cases} 
      Normal & $si $  N_{s}(U) \leq \theta \\
      Anomalo & $si $  N_{s}(U) > \theta \\ 
   \end{cases}
\end{equation}


El umbral de detección es un parámetro que se calcula de forma experimental. Durante la pruebas se busca conseguir un valor $\theta$ con el cual se pueda obtener la proporción óptima entre las detecciones de anomalías correctas y los falsos positivos. 

No obstante, la fórmula presentada en el apartado \ref{eq:Ns1} posee algunos inconvenientes ya que no se estipula el caso en el que existan problemas de entrenamiento insuficiente que será presentado en la sección o que los diferentes vectores B, tengan diferente número de observaciones. El segundo caso provoca un inconveniente a la hora de realizar la evaluación ya que el acumulado de las probabilidades de los vectores B estarán ligados a la longitud de los mismos. Para solucionar dicho problema se utilizara la propuesta de Estévez [Estévez-Tapiador 2004a], en el que se establece como factor de compensación la probabilidad de observación para de esta manera obtener una probabilidad normalizada. Esta sería la siguiente expresión:

\begin{equation}\label{eq:sumB}
\varepsilon_{0} = E[B] = \frac{1}{M}\sum_{i=1}^{M}b_{i}
\end{equation}

Por otra parte, el entrenamiento insuficiente es un problema que se presenta cuando se toma una palabra durante la evaluación que no pertenezca al conjunto de palabras observadas durante el entrenamiento en el modelo. Esto puede ser debido a que la palabra es una palabra anómala o que no se entrenó lo suficiente al sistema. En este caso, la solución que se le aplicará a dicho problema será la de asignar un valor fijo de probabilidad muy baja a estos casos ( poov($S_{i}$): probabilidad fuera de vocabulario). Cada uno de los estados del modelo poseerá valores independientes de probabilidades que serán utilizados una vez se detecte este problema.

El otro caso en el que se utilizarán los valores de poov($S_{i}$), es cuando la probabilidad de una palabra que está en el vocabulario es inferior al poov($S_{i}$), es decir:


\begin{equation}\label{eq:Pqtot}
p_{qtot} = 
	\begin{cases} 
      b_{qtot} & $si $  o_{t} \in O $ y $ b_{qtot} > p_{oov}(q_{t})\\
      p_{oov} & $en otro caso$ \\ 
   \end{cases}
\end{equation}


Entonces, la nueva expresión del índice de anormalidad que toma en cuenta todas las consideraciones nombradas con anterioridad, es la siguiente:

\begin{equation}\label{eq:Ns}
N_{s} = 
	\begin{cases} 
      -Tlog(\varepsilon_{0})\sum_{t=1}^{T}log(p_{qtot}) & $si $  q_{t} = S_{f} \\
      \infty & $si $  q_{t} \neq S_{f} \\ 
   \end{cases}
\end{equation}

\subsection*{Entrenamiento}

Dada una secuencia de observaciones 0 = o1,o2,...,oT ,y  su correspondiente secuencia de estados, Q = q1,q2,...,qT.

El entrenamiento requiere de un conjunto considerable de observaciones  junto con los estados asociados a cada una de ellas.

Por lo tanto, se considerara un conjunto de entrenamiento $\omega$ de L pares de secuencias de estados

\begin{equation}
\omega = {(O^{1},Q^{1}),(O^{2},Q^{2}),...,(O^{L},Q^{L})}
\end{equation}

tal que:

\begin{equation}
\begin{aligned}
O^{i} = {o_{1}^{i},o_{1}^{i},...,o_{Ti}^{i}} \\
Q^{i} = {q_{1}^{i},q_{2}^{i},...,q_{Ti}^{i}}
\end{aligned}
\end{equation}

Entonces, para crear el vocabulario del modelo de normalidad solo se tendrá que hacer un recuento de las frecuencias de aparición relativas de los símbolos y los estados, es decir:

\begin{equation}
\theta_{Sk} = \bigcup\limits_{i=1}^{L} {o_{j}^{i}|q_{j}^{i} = S_{k}}
\end{equation}

Por otra parte, para calcular la probabilidades de observación se utilizaría la siguiente expresión:

\begin{equation}\label{eq:entrenamiento}
b_{ij} = \frac{\sum_{s=1}^{L}\sum_{t=1}^{Ts}\delta(o_{t}^{s} = v_{i,j} ,q_{t}^{s} = s_{i} )}{\sum_{s=1}^{L}\sum_{t=1}^{Ts}\delta(q_{t}^{s} = s_{i}) }
\end{equation}

Donde, $\delta$ es una función que retorna 1 si todos sus argumentos son verdaderos y 0 cuando son falsos.

\subsection*{Normalización de los URIs}
\label{sec:normalizacion}

La normalización consiste en tomar el URL en el formato en el que viene y codificarlo al formato de tipo UTF-8. Este paso evitará considerar palabras o frases iguales que estén escritas en formatos diferente como elementos distintos. 

Un ejemplo sencillo de lo que se haría en la normalización del sistema sería el siguiente:

Supongamos que al servidor de tipo HTTP recibe dos peticiones de tipo GET con los siguientes URIs:
https://192.168.0.23?q=security+network
https://192.168.0.23?q=security\%20network

Ambos URIs poseen la misma información, sin embargo, el espacio en blanco esta codificado de manera diferente en ambos casos. Si no se lleva a cabo la normalización, la palabra "security+network" y "security\%20network" serían tomadas como dos palabras totalmente diferentes por el sistema. En este caso, la función de normalización se encargaría de traducir el primer URI al formato UTF-8 para que no exista este problema. Por lo tanto,la frase "security+network" pasaría a ser  "security\%20network".


\subsection*{Segmentación de los URIs}
\label{sec:delimitadores}

La idea de la segmentación de URI es considerar una serie de delimitadores, que delimitan áreas especificas dentro del URI.

Los delimitadores que se tomarán en cuenta para segmentar serán los asociados al estándar del protocolo HTTP. Estos son los siguientes:

\begin{itemize}
\item D1 = ://, delimitador de protocolo.
\item D2 = /, delimitador de recursos.
\item D3 = ?, delimitador de parámetros.
\item D4 = =, delimitador de asignación de atributos.
\item D5 = \&, delimitador entre parametros.
\item D6 = ASCII 32, delimitador de fin de recurso
\end{itemize}

A modo ilustrativo, se utilizará un ejemplo concreto con un URI para mostrar como funciona la segmentación en el sistema.

Supongamos que tenemos el siguiente URI:

http://159.90.9.166/consulta/?manifestacion=Pinturas+Rupestres

Entonces, la función que se encarga de segmentar tomará dicho URI y lo segmentará en las siguientes partes:

\begin{itemize}
\item http : Segmento correspondiente al protocolo del URI. 
\item 159.90.9.166 : Segmento correspondiente al ``host'' del URI. 
\item consulta : Segmento correspondiente al recursos del URI.
\item manifestacion : Segmento correspondiente al atributo del query.
\item Pinturas\%20Rupestres : Segmento correspondiente al valor del query.  
\end{itemize}

















 




\chapter{Marco técnologico}
\label{capituloTecnologico}

\section{Herramientas utilizadas}

\subsection{Bro}

Bro es un software ``open source'' cap\'az de analizar a detalle las actividades que ocurren en la red. Se utiliza principalmente para inspeccionar todo el tráfico y detectar signos de actividad sospechosa. Sin embargo, Bro soporta una amplia gama de tareas de análisis de tráfico incluso fuera del dominio de seguridad, incluyendo mediciones de rendimiento y ``trouble-shouting''.\cite{Bro}

Este software, posee un conjunto extenso de $"logs"$ que registran las actividades que se observan en la red. Estos $"logs"$ almacenan información de la capa de aplicación, como por ejemplo, los URIs y las respuestas del servidor en una sesión de tipo HTTP, o las solicitudes y respuestas DNS en una sesión DNS. Bro escribe todos estos datos en archivos con formatos bien predeterminados con la intención de facilitar el procesamiento de la información almacenada por programas externos. No obstante, el usuario tiene la posibilidad de elegir el formato en el que se almacenan los datos.

Por otra parte, Bro incluye también un lenguaje de $"scripting"$, Turing completo y orientado a eventos, cuya funci\'on es extender y personalizar las funcionalidades primarias que otorga el mismo. Este lenguaje de $"scripting"$ posee una biblioteca estándar. 

Así mismo, el lenguaje de Bro posee diversos ``frameworks'' que facilitan el trabajo al momento de programar nuevas funcionalidades. Los ``frameworks'' utilizados en el sistema implementado fueron: el ``Input Framework'' y el ``Logging Framework''.
El ``Input Framework'' permite leer de manera mas cómoda y eficiente la información que se encuentra dentro de los $"logs"$ del sistema.
Este $"framework"$ posee dos modalidades. La primera modalidad consiste en leer la información que se encuentra dentro de los $"logs"$ para luego almacenarla en una estructura de datos de tipo $"table"$. La segunda modalidad lee la información y la envía a un evento anteriormente programado.
Por otra parte, el ``Logging Framework'', tiene como tarea facilitar el manejo y la escritura de los $"logs"$. Este, permite crear, escribir datos de forma organizada y filtrar información de los $"logs"$.

Como se mencionó con anterioridad, el lenguaje de $"scripting"$ de Bro es un lenguaje orientado a objetos. Sin embargo, es importante recalcar, que el núcleo o motor principal de Bro es el motor de eventos, el cual reduce la información entrante de la red, en eventos de alto nivel. Por ejemplo, si se recibe una solicitud de tipo HTTP, Bro se encargará de convertir la misma en un evento de tipo $"http\_request"$ en donde se almacenará información como: la dirección IP, los puertos involucrados, el URI y la versión de tipo HTTP en uso.

A su vez, los eventos creados por Bro a partir de la información entrante de la red son enviados a los manejadores de eventos escritos en el lenguaje de $"scripting"$ de Bro. Estos manejadores administran dichos eventos a tráves de una cola ordenada.

En Bro hay eventos primitivos, como es el caso de $"http\_request"$ --evento generado por las solicitudes tipo HTTP -- , $"dns\_request"$, --evento generado por las solicitudes tipo DNS-- , $"bro\_init"$ -- evento generado al momento en el que Bro inicializa-- y $"bro\_done"$ --evento que se genera cuando se culminan de realizar todas las tareas que se deben realizar--. Sin embargo, este software brinda la capacidad a los programadores de implementar eventos propios. Estos eventos son un tipo especial de función que poseen las características de: no retornar valores, poder ser programados para una ejecución posterior y poseer una prioridad asociada y configurable que determine el orden en el que serán ejecutados.

Un ejemplo de la implementación de un evento en el lenguaje de Bro sería la siguiente:

\begin{verbatim}
event myevent(s: string)
	{
	print "myevent", s, n;
	}
\end{verbatim}

Así como se pueden escribir eventos en Bro, también se pueden implementar funciones haciendo uso de la palabra reservada $"function"$. Una función en Bro luciría de la siguiente manera":

\begin{verbatim}

function factorial(n: count): count
    {
    if ( n == 0 )
        return 1;
    else
        return ( n * factorial(n - 1) );
    }

\end{verbatim}

Por otra parte, el lenguaje de Bro además de permitir implementar eventos y funciones, posee diversos tipos y estructura de datos que facilitan el manejo de la información. 
El conjunto de tipos de datos que posee el mismo se basa en los tipos de datos comunes que encontramos en la mayoría de los lenguajes de programación como: el $"int"$, que representa los números enteros de 64 bits, $"double"$ que representa los números flotantes de 64 bits, $"bool"$ para los booleanos y $"count"$  para los números enteros sin signo de 64 bits, y en un grupo de tipos de datos  mas especifico como: el $"addr"$ que representa direcciones ip, $"port"$ para los puertos de la capa de transporte, $"subnet"$ para las mascaras de subred, $"time"$ para almacenar datos que representen tiempo, $"interval"$ para representar intervalos de tiempo y $"pattern"$ para las expresiones regulares.

Las estructura de datos que permite el lenguaje son: los conjuntos ("set"), las tablas ("table"), vectores ("vector") y registros ("record").

\subsection{GitHub}

\subsection{Wireshark}

\section{Estación de trabajo}
\chapter{Diseño y arquitectura del sistema}
\label{capitulo3}
\lhead{Capítulo 3. \emph{Diseño del sistema}}

En este capitulo se explicará la arquitectura, el diseño y el modo en el que el IDS basado en SSM interactuara con Bro. Además, se detallaran las salidas y los datos de configuración que va a considerar el sistema para su correcto funcionamiento vas a considerar. Este capitulo tiene la intención de mostrar la visión general del modelado del sistema que se realizo a partir de las bases teóricas.

\section{Arquitectura del sistema}

La arquitectura del detector de intrusiones que se muestra en el presente trabajo se basa en una arquitectura modular la cual está conformada por tres modulo. Un modulo para realizar la segmentación de los URIs, otro  para realizar la evaluación y un tercero para realizar el entrenamiento y de esta forma crear un modelo de normalidad.

A grandes rasgos, el modulo de segmentación se encargará de tomar el URI que proviene de la solicitud de tipo GET o HEAD que se le hace al servidor HTTP, normalizarlo y segmentarlo siguiendo las especificaciones del RFC (insertar nombre del RFC) .

Por otra parte, el modulo de evaluación se encargará de evaluar la probabilidad de generación de cada uno de los segmentos del URI generados por el modulo de segmentación para al final decidir si el URI de la solicitud enviada al servidor de tipo HTTP es anómalo o no.

Por último, el modulo de entrenamiento será el encargado de crear el modelo de normalidad del sistema. Para esto, el sistema recibirá solicitudes libres de ataques e irá calculando la probabilidad de aparición de cada una de las palabras que aparecen en las mismas.

Tanto el modulo de evaluación, como el modulo de entrenamiento son dependientes del modulo de segmentación ya que requieren de los segmentos de URI generados por eso para realizar su trabajo. 

La arquitectura del sistema queda detallada en la figura \ref{fig:arquitectura}.

\begin{figure}[tb]
\begin{center}
\includegraphics[width=3in]{arquitectura.png}
\caption{Arquitectura del sistema.}
\label{fig:arquitectura}
\end{center}
\end{figure}

\begin{itemize}
\item Modo evaluación: En este modo de operación solo se activaran los módulos de segmentación y evaluación del sistema. A manera general, esta modalidad, se encargará de recibir peticiones de tipo HTTP/GET, extraer su URI y segmentarlos a tráves del modulo de segmentación para luego evaluar si  el mismo es anómalo o no, haciendo uso del modulo de evaluación. El funcionamiento de esta modalidad queda detallado en la figura \ref{fig:modoSistema}.
\item Modo entrenamiento ``Online'': En este modo de operación solo trabajaran los módulos de segmentación y entrenamiento. Esta modalidad es un tipo de entrenamiento en donde se toman peticiones de tipo HTTP/GET y se segmenta el URI que se encuentran en las mismas haciendo uso del modulo de segmentación. Una vez realizado esto, el modulo de entrenamiento tomara los segmentos arrojados por el modulo de segmentación, calculara la probabilidad de aparición de los mismos para de esta manera ir modificando un modelo de normalidad previamente establecido, es decir, la salida de este modo de entrenamiento sera la de el modelo previamente establecido con ciertas modificaciones realizadas a partir de las observaciones hechas. El funcionamiento de esta modalidad queda detallado en la figura \ref{fig:modoSistema}..
\item Modo entrenamiento ``Offline'': Esta modalidad del sistema en análoga al modo de entrenamiento ``Online''. El único aspecto que diferencia a ambas modalidades es que cuando el sistema funciona en modo ``Offline'' no se toma en cuenta, ni se modifica un modelo de normalidad previamente construido. La salida de este modo de entrenamiento sera un modelo de normalidad construido desde cero.El funcionamiento de esta modalidad queda detallado en la figura \ref{fig:modoSistema}.
\end{itemize}

\begin{figure}[tb]
\begin{center}
\includegraphics[width=\linewidth]{modoOperacion.jpeg}
\caption{Modo evaluación (a), modo entrenamiento ``Online'' (b), modo entrenamiento ``Offline'' (c).}
\label{fig:modoSistema}
\end{center}
\end{figure}

\section{Módulos}

En esta sección se describirán las tareas de cada uno de los módulos que conformar la arquitectura detallada en la figura \ref{fig:arquitectura}, el modo en el que se subdividieron dichas tareas, los datos de entrada y salida, y la interacción que existe entre los mismos.

\subsection{Modulo de segmentación}

El modulo de segmentación es un elemento clave dentro de la construcción del IDS basado en SSM, es por eso que su buen diseño e implementación es importante para el buen funcionamiento del sistema. Este se encarga, como se menciono anteriormente,de tomar el URI que proviene de la solicitud de tipo GET o HEAD que se le hace al servidor HTTP capturado por Bro, normalizarlo y segmentarlo siguiendo las especificaciones del RFC (insertar nombre del RFC) 

Es evidente entonces, que este modulo consta de dos funcionalidades fundamentales: la normalización de los URIs y la segmentación.  La normalización se encargará de tomar el URI de las peticiones HTTP entrantes y codificarlo a formato UTF-8. Normalizar es un paso importante dentro del sistema  ya que se estandarizar la forma en las que están escritos los URIs facilita tanto la evaluación como el entrenamiento en el sistema. Por otra parte, el output arrojado por esta función de normalización será tomado por la de segmentación, quien a su vez se encargará de segmentar el URI de la forma en la que se explica en la sección \ref{sec:delimitadores}, es decir, el URI se dividirá en las diferentes partes estipuladas en el RFC 3986: el ``host'', la ruta, los argumentos, los valores y el ``fragment''.

 En las base teórica, la segmentación de los URIs se realiza mediante un autómata que se encarga de reconocer (realizar un análisis sintáctico) y evaluar en cada uno de sus estados la probabilidad de generación de cada uno de los segmentos.  No obstante, en la función de segmentación del sistema implementado, esta tarea se modeló mediante un analizador sintáctico que hace uso de una gramática (libre de contexto) de atributos que genera el mismo lenguaje que reconoce el autómata presentado en la figura ~\ref{fig:automata}, es decir, el lenguaje de los URI.

En la figura (insertar nombre de la figura) se puede apreciar el diagrama de bloques que refleja el funcionamiento del módulo de segmentación.

\begin{figure}[tb]
\begin{center}
\includegraphics[width=3in]{segArquiCompleta.jpeg}
\caption{Diagrama de bloques del módulo de segmentación.}
\label{fig:modoSistema}
\end{center}
\end{figure}

*************************************************************************8
Dentro de este modulo se pueden contemplar dos grandes funcionalidades: la normalización y la segmentación.

En la normalización se encargará de traducir los URIs entrantes al formato UTF-8, mientras que, la segmentación tendrá el trabajo de dividir los URIs en segmentos. 

``En el capítulo 3, se pudo apreciar la arquitectura del módulo de segmentación. Además, se pudo observar que el mismo consta de dos funcionalidades fundamentales: la normalización de los URIs y la segmentación. A grandes rasgos, la normalización se encargará de tomar el URI de las peticiones HTTP entrantes y codificarlo a formato UTF-8, el output arrojado por esta función será tomado por la de segmentación, quien a su vez se encargará de segmentar el URI de la forma en la que se explica en la sección \ref{sec:delimitadores}, es decir, el URI se dividirá en las diferentes partes estipuladas en el RFC 3986: el host, el path, los argumentos, los valores y el fragment''

``El problema a resolver, presenta dos tareas fundamentales: segmentar y realizar un análisis sintáctico. En programación cuando se tiene un conjunto de datos de entrada (en este caso serían los URIs), y se desea realizar un análisis sintáctico sobre el mismo, lo más común es implementar un analizador sintáctico o parser.

El análisis realizado por un analizador sintáctico “es el proceso en el cual se estudia la manera en la que una cadena de caracteres terminales es generado por una gramática.”  Compiler, principle, techniques and tools."

Por otra parte, existen varios tipos de gramáticas, las gramáticas libre de contexto y las gramáticas regulares. Las que se suelen utilizar en un analizador sintáctico son las gramáticas libres de contexto. Una gramática libre de contexto “en el contexto del lenguaje natural, sería un conjunto de reglas que son utilizadas para construir o validar oraciones” “Formal Languages and Automata Theory” de D. Goswami and K. V. Krishna. 

De manera más formal, según la definición 3.1.1 de: “Formal Languages and Automata Theory” de D. Goswami and K. V. Krishna. Se puede decir que una gramática libre de contexto es una  cuádrupla G = (N, $\sum$, P, S), donde:
\begin{enumerate}
\item N es un conjunto de no-terminales,
\item $\sum$ es un conjunto de terminales. Los terminales son elementos fundamentales en el lenguaje generado por la gramática.
\item S $\in$ N es el símbolo de inicio.
\item P es un subconjunto finito de N x V * llamado el conjunto de reglas de producciones. Aquí, V = N U $\sum$."
\end{enumerate}

Entonces, para realizar el analizador sintáctico se requiere construir una gramática libre de contexto que genere el mismo lenguaje que reconoce el autómata presentado en la imagen ~\ref{fig:automata}, es decir, el lenguaje de los URI.

Antes de presentar la gramática se mostrarán los tokens que utilizará la misma como elementos terminales. Un token “es la cadena de caracteres más corta que posee algún significado” Scott, Compilers.''




A continuación se explicará con mas detalle ambas funcionalidades. 

``La normalización es un paso importante dentro del modulo de segmentación ya que se encarga de estandarizar la forma en las que están escritos los URIs para de este modo facilitar tanto la evaluación como el entrenamiento en el sistema.''

\subsection{Modulo de evaluación}
\label{sec:evaluacion}

Este modulo, como se mencionó con anterioridad es el encargado de evaluar la probabilidad de generación de cada uno de los segmentos del URI dado un modelo de normalidad. Una vez calculadas estas probabilidades, el modulo se encargará de calcular un índice de anormalidad del URI mediante el uso de las formulas descritas en la sección ~\ref{subsec:exprIndice} para luego compararlo con un parámetro $\theta$ y de este modo saber si el URI posee alguna anormalidad o no. 


\subsection{Modulo de entrenamiento}\label{sec:entrenamiento}

El modelo de normalidad es uno de los aspecto importantes dentro del IDS basado en SSM ya que a partir de la información que este almacena se podrá decidir si un URI es anómalo o no dependiendo de la similitud que posea este con los datos que almacena el modelo. Por esta razón, realizar un modelo de normalidad apropiado es un aspecto  importante para que el módulo de evaluación haga detecciones de intrusiones más certeras.

El módulo encargado de elaborar los modelos de normalidad será el módulo de entrenamiento. A grandes rasgos, este se encargará de recibir solicitudes libres de ataques e irá calculando la probabilidad de aparición de cada una de las palabras que aparecen en los URIs de las mismas, para finalmente dar al sistema una lista de palabras observadas junto con su probabilidad de aparición.

En esta sección se explicará con detalles el funcionamiento y las expresiones utilizadas por el módulo de entrenamiento para construir el modelo de normalidad del sistema. 

``Este módulo de entrenamiento será el encargado de armar el vocabulario del módulo de normalidad del sistema. Como se mencionó con anterioridad, este módulo tomará un conjunto de solicitudes libres de ataques, para luego calcular la probabilidad de aparición de las palabras en el conjunto de observaciones totales de cada uno de los estados. En términos más formales, la  tarea que realiza este módulo se puede modelar de la siguiente forma:''
\chapter{Detalles de implementación}

En el capítulo anterior se pudo apreciar la arquitectura del IDS que se desea implementar haciendo uso del lenguaje de scripting de Bro. Allí se pudo observar que el mismo está constituido básicamente por tres módulos fundamentale: un módulo de segmentación que se encarga de normalizar y segmentar el URI. Un módulo de evaluación que se encargará de evaluar la probabilidad de generación de un URI dado un modelo de normalidad para así clasificar el mismo como normal o anormal y un módulo de entrenamiento cuyo trabajo será crear modelos de normalidad dados un conjunto significante de URI sin alguna anormalidad.

A continuación se explicará las estructuras de datos utilizadas y como se realizó la implementación de cada uno de los módulos mencionados anteriormente haciendo uso de las herramientas presentadas por el lenguaje de scripting de Bro.

\label{capitulo4}
\lhead{Capítulo 4. \emph{Detalles de implementación}}
\section{Implementación del modulo de segmentación}

A continuación se explicará de manera detallada la implementación del módulo de segmentación y las estructuras de datos utilizadas en el mismo. Este módulo forma parte de los tres que conforman el sistema de detección de intrusiones basado en SSM. En el capítulo anterior se pudo apreciar la arquitectura del módulo de segmentación. Además, se pudo observar que el mismo consta de dos funcionalidades fundamentales: la normalización de los URIs y la segmentación. A grandes rasgos, la normalización se encargará de tomar el URI de las peticiones HTTP entrantes y codificarlo a formato UTF-8, el output arrojado por esta función será tomado por la de segmentación, quien a su vez se encargará de segmentar el URI de la forma en la que se explica en la sección (inserte número de la sección), es decir, el URI se dividirá en las diferentes partes estipuladas en el RFC (insertar número del RFC): el host, el path, los argumentos, los valores y el fragment. Estos segmentos obtenidos luego ser almacenados en una estructura de datos global para que de esta forma, la información obtenida pueda ser utilizada por el resto de los módulos.

La implementación de la función de normalización que se explicará a continuación está basada en el uso de una hashtable con todos los elementos de una encoding table de tipo UTF-8. Por otra parte, la función de segmentación que teóricamente esta basada en un autómata que reconoce el lenguaje de los URI se implementará tomando la inspiración de una gramática que genere el mismo lenguaje que reconoce el autómata (el lenguaje de los URIs). Esto se puede realizar ya que las gramáticas poseen el mismo nivel de expresividad que un autómata, como se mostrará en la secciones próximas. Además, las herramientas presentadas por el lenguaje de scripting Bro facilitan la implementación de un segmentador inspirado en una gramática.

\subsection{Estructura de datos}
En el módulo de segmentación existen dos estructuras de datos fundamentales. Una es utilizada en la función de normalización y la otra en la de segmentación.

\textbf{Función de normalización}

La estructura de datos utilizada en la función de normalización es un hashtable que contiene los elementos de un encoding table de tipo UTF-8, en donde las claves de la tabla serían los elementos de tipo UTF-8 y los atributos de las mismas corresponderían a los caracteres sin alguna codificación, ya que lo que se quiere es un diccionario que mapee los elementos de tipo UTF-8 con sus respectivos caracteres lo mas rapido posible.

Se utilizó un hash table para implementar este diccionario debido a que es ampliamente conocido que este tipo de tablas propocionan mucha eficiencia en el tiempo de búsqueda de sus elementos, como lo explica Cormen en su libro: "Introduction to Algoriths":

“Una hash table es una estructura de datos efectiva para implementar diccionarios. A pesar de que la búsqueda de un elemento en una hash table puede tomar el mismo tiempo de búsqueda que una lista enlazada - O(n)/tiempo en el peor de los casos - en la práctica, el mapeo funciona extremadamente bien.  Bajos suposiciones razonable, el tiempo de búsqueda promedio de una hash table es de O(1)”.

Para implementar este tipo de estructura se utilizó el tipo de dato “table” otorgado por el lenguaje de scripting de Bro. Un ejemplo concreto de como luce la estructura de dato implementada en Bro se mostrará a continuación:

\begin{verbatim}
global encoding : table[string] of string = {    ["%21"] =    "!"     ,
                                                ["%22"] =    "”"    ,                                        
                                                ["%23"] =    "#"    ,
                                                ["%24"] =    "$"    }
\end{verbatim}

La funcionalidad de esta tabla de codificación será explicada en la sección (insertar sección de implementación de normalización)


\textbf{Función de segmentación}

Por otra parte, la estructura de datos utilizada en la función de segmentación de este módulo es un registro en el cual se almacenan todos los segmentos que se originan a partir de la segmentación de un URI, el URI sin segmentar, un booleano que informa si el URI segmentado sigue con la sintaxis establecida en el RFC (insertar número del RFC), y el número de estados que fueron visitados en el autómata presentado en el modelo teórico (insertar número de imagen) para reconocer al mismo .


En Bro existen las palabras claves “record” y “type” y se utilizan de forma similar en las que se emplean las palabras claves “typedef” y “struct” en C. Con estas palabras, Bro permite combinar nuevos tipos de datos y crear tipos de datos compuestos para adaptarse a las necesidades de la situación. A continuación, se mostrará un ejemplo de un tipo de dato compuesto escrito en Bro.

Extraído de la documentación de Bro.

\begin{verbatim}
type Service: record {
    name: string;
    ports: set[port];
    rfc: count;
};
\end{verbatim}

Cuando se combina la palabra clave “type”, el registro puede generar un tipo de dato compuesto.

Se escogió un registro para almacenar la información ya que el lenguaje de scripting de Bro solo presenta como alternativa el tipo de dato “record” para crear un tipo de dato compuesto. Es necesario crear este tipo de dato ya que se requiere una estructura que almacene los diferentes tipos de datos de los que se hablaron anteriormente. 

El registro creado esta conformado por los siguientes campos:
\begin{itemize}
\item uri: Es una variable de tipo string que almacena el URI sin segmentar. Este campo del registro será utilizado por el módulo de entrenamiento al momento de escribir los logs correspondientes.
\item host: Es una variable de tipo hash table cuyas claves son número y sus valores  son de tipo string. En esta sección se almacenan los segmentos del URI correspondiente al host.
\item path: Es una variable de tipo hash table cuyas claves son número y sus valores  son de tipo string. En esta sección se almacenan los segmentos del URI correspondiente al path.
\item query: Es una variable de tipo hash table cuyas claves y sus valores son de tipo string. En esta sección se almacenan los atributos y los valores del query del URI en caso de que el mismo posea. Los atributos y los valores se almacenarán en una hash table como se mencionó con anterioridad, en donde la clave del mismo serán los atributos del query y en donde los valores de las misma serán los valores del query del URI.
\item fragment: Es una variable de tipo string que almacenará el fragment del URI en caso de poseerlo.
\item número de estados: es una variable de tipo entero que almacena el número de estados del autómata que se emplea en el módelo teórico fueron visitados para reconocer el URI. Este campo del registro será utilizado por el módulo de evaluación.
\item uri correcto: es una variable de tipo booleano que indica si el URI está sintácticamente correcto o no.
\end{itemize}

El uso de esta estructura y el de cada uno de sus campos se explicara a detalle en las secciones siguientes.

\subsection{Implementación del automata de reconocimiento de URIs}
Explicar lo que se debe implementar.
Explicar que los automatas son tan potentes y tienen el mismo nivel de expresividad que una grámatica.
Explicar que la implementacion se baso en modelar una gramatica con Bro a partir de funciones.
EXplicar mas o menos como es el funcionamiento.
Explicar como se guardan los parametros.
\subsection{Implementación del la normalización de los URIs}
Explicar que se utilizó la encoding table
\section{Implementación del modulo de evaluación}
\subsection{Estructura de datos}
\subsection{Lectura del modelo de normalidad en Bro}
\subsection{Evaluación de las probabilidades de los URI}
\section{Implementación del modulo de entrenamiento}
\subsection{Estructura de datos}
\subsection{Entrenamiento Online}
\subsection{Entrenamiento Offline}
\subsection{Escritura del modelo de normalidad en Bro}
\chapter{Evaluación y pruebas}
\label{capitulo5}
\lhead{Capítulo 5. \emph{Evaluación y pruebas}}
\section{Base de datos}
\subsection{Base de datos normales}
\subsection{Base de datos anomalas}
\section{Experimentación}
\subsection{Pruebas funcionales}
\subsection{Pruebas operativas}

\chapter*{Conclusiones y Recomendaciones}
\label{conclusiones}

% Incluir recomendaciones para trabajos futuros
\addcontentsline{toc}{chapter}{Conclusiones}

En este capítulo se presentarán las conclusiones y los aportes realizados por este proyecto. Así mismo, se mostrará una lista de recomendaciones, que expone un conjunto de ideas de futuras mejoras que se pueden aplicar al proyecto.

\section*{Conclusiones}

En el presente proyecto se realizaron todos los objetivos planteados de manera satisfactoria. En primer lugar, se realizó un estudio de SSM (``Stochastic Structural Model'') \cite{ssm} y todos sus conceptos asociados. Una vez realizado este investigación, se diseñó una arquitectura modular del sistema, en la cual se propuso la construcción de tres módulos: uno para capturar, filtrar y segmentar los paquetes de tipo \textit{GET} del protocolo \textit{HTTP}; otro módulo que permite evaluar el índice de anormalidad de los URI y un tercer módulo para estimar los modelos de normalidad. Luego de modelar la arquitectura del sistema, se implementó cada una de las funciones del mismo haciendo uso de la herramientas y del lenguaje de ``scripting'' de Bro, con la intención de obtener un IDS basado en SSM que puede ser utilizado por servicios web en producción. Por último, se realizaron pruebas tanto funcionales como operativas. 

Los aportes realizados por este proyecto son los siguientes:

\begin{enumerate}
\item Se ha implementado un sistema que realiza detección de intrusiones en los servidores web.
\item Se ha implementado un sistema con tres funcionalidades operativas: una para realizar detección de intrusiones y dos para realizar modelos de normalidad de los servicios web.
\item Se han construido modelos de normalidad de servicios web que podrían ser utilizados para la detección de intrusiones.
\end{enumerate}

\section*{Recomendaciones}

\begin{enumerate}
\item Calcular de manera experimental los valores óptimos de $\theta$ y los de probabilidad fuera de vocabulario que generen la menor cantidad posible de falsos positivos, haciendo uso de bases de datos mas extensas.
\item Agregarle al IDS un conjunto de firmas de ataques conocidos aplicando la técnica propuesta en \cite{firmas}.
\item Probar el sistema en un entorno de producción.
\item Ampliar el sistema para que funcione en otros protocolos de la red.
\end{enumerate}


% El estilo de la bibliografía es AAAI, definido en el archivo aaai.bst.
\begin{thebibliography}{99}
\bibitem{pwc} 
PWC: PricewaterhouseCoopers,
\\\texttt{\url{https://www.pwc.com/gx/en/services/advisory/forensics/economic-crime-survey/cybercrime.html}}

\bibitem{salario}
Walters R. (2017), \textit{Salary Survey}, disponible en: \url{https://www.robertwalters.co.uk/content/dam/salary-survey-2017.pdf?webSyncID=6d3bc948-9837-8e7b-91d2-a51713c993b7&sessionGUID=0b156ce4-a2a8-3d94-8cd0-15640ae068ac}

\bibitem{IDS}
H.   Debar (2000), \textit{An Introduction to Intrusion-Detection Systems},  IBM   Research,   Zurich   Research   Laboratory,Ruschlikon, Switzerland.

\bibitem{rfcHTTP}
RFC 7230

\bibitem{rfcURI}
RFC 3986

\bibitem{automata}
Linz P. (2012) \textit{An Introduction to FORMAL LANGUAGE an AUTOMATE}, quinta edición, pp 36.

\bibitem{automata2}
Mishra K.L.P. , Chandrasekaran N. (2008), \textit{THEORY OF COMPUTER SCINCE Automata, Language and Computation},tercera edición, pp 71-72.

\bibitem{automataFinito}
V. Aho A., S. Lam M., Sethi R., D. Ullman J. (2007),\textit{Compilers, Principles, Techniques, \& Tools}, segunda edición,pp 147-150.

\bibitem{markov}
Ching W., Ng M. K. (2006), \textit{Markov Chains: Models, Algorithms and Applications}, pp 1-4.

\bibitem{ejemploMarkov}
Haggstrom (2002) \textit{Finite Markov Chains and Algorithmic Applications}, Lon-
don Mathematical Society, Student Texts 52, Cambridge University Press,Cambridge, U.K.

\bibitem{ssm}
Pedro García-Teodoro.; Jesús E. Díaz-Verdejo; Juan M. Tapiador; Rolando Hernandez-Salazar
Automatic Generation of HTTP Intrusion Signatures by Selective Identification of Anomalies
Computers \& Security, Vol. 55, pp. 159-174, 2015, ISSN: 0167-4048 

\bibitem{tesisMexico}
Rolando Salazar Hernández, Sistema de detección de intrusos mediante modelado de URI. Tesis doctoral. Universidad de Granada. Director: Jesús E. Díaz Verdejo, 02/02/2016

\bibitem{Bro}
\textit{Bro Introduction}, disponible en : \url{https://www.bro.org/sphinx/intro/index.html}

\bibitem{httpKross}
Kurose, J. F., \& Ross, K. W. (2013). \textit{Computer networking: a top-down aproach}. Pearson

\bibitem{Cormen} Cormen H. C., Leiserson E. C., Rivest L. R., Stein C. (2009).Introduction to algorithms, tercera edición, pp 153-155.

\end{thebibliography}

%\label{Bibliography}
%\bibliography{bibliografia}
%\lhead{\emph{Bibliografía}}
%\bibliographystyle{aaai}
%\addtocontents{toc}{\vspace{2em}}

% Apéndices
\appendix
\chapter{Estructuras de datos del sistema}
\label{apendiceA}
\lhead{Apéndice A. \emph{Estructuras de datos del sistema}}

% En los apéndices se incluye cualquier información que no sea esencial para la
% comprensión básica del trabajo, pero provea ejemplos y casos de estudio
% extendidos que permitan un análisis más exhaustivo.


\section*{Estructura de Datos del Módulo de Segmentación}

En el módulo de segmentación existen tres estructuras de datos fundamentales. Una es utilizada en la función de normalización y la otras dos en
la de segmentación.

\subsection*{Normalización}
\label{ssec:estructuraNormalizacion}

La estructura de datos utilizada en la función de normalización es un hash table que contiene los elementos de un encoding table de tipo UTF-8, en donde las claves de la tabla serían los elementos de tipo UTF-8 y los atributos de las mismas corresponderían a los caracteres sin alguna codificación. Con esta tabla, se quiere implementar un diccionario que mapee los elementos de tipo UTF-8 con sus respectivos caracteres lo mas rapido posible.

Se utilizó un hash table para implementar este diccionario debido a que es ampliamente conocido que este tipo de tablas propocionan mucha eficiencia en el tiempo de búsqueda de sus elementos \cite{Cormen}.

Para implementar este tipo de estructura se utilizó el tipo de dato ``table'' otorgado por el lenguaje de scripting de Bro.

\subsection*{Función de segmentación}
\label{ssec:estructuraSegmentacion}

Por otra parte, la estructura de datos utilizada en la función de segmentación de este módulo es un registro en el cual se almacenan todos los segmentos que se originan a partir de la segmentación de un URI, el URI sin segmentar, un booleano que informa si el URI segmentado sigue con la sintaxis establecida en el RFC 3896, y el número de estados que fueron visitados en el autómata presentado en el modelo teórico.
 
Se escogió un registro para almacenar la información ya que el lenguaje de ``scripting'' de Bro solo presenta como alternativa el tipo de dato ``record'' para crear un tipo de dato compuesto. 

El registro creado esta conformado por los siguientes campos:
\begin{itemize}
\item uri: Es una variable de tipo ``string'' que almacena el URI sin segmentar. Este campo del registro será utilizado por el módulo de entrenamiento al momento de escribir los logs correspondientes.
\item host: Es una variable de tipo ``hash table'' cuyas claves son número y sus valores  son de tipo string. En esta sección se almacenan los segmentos del URI correspondiente al host.
\item path: Es una variable de tipo ``hash table'' cuyas claves son número y sus valores  son de tipo string. En esta sección se almacenan los segmentos del URI correspondiente al path.
\item query: Es una variable de tipo ``hash table'' cuyas claves y sus valores son de tipo string. En esta sección se almacenan los atributos y los valores del query del URI en caso de que el mismo posea. Los atributos y los valores se almacenarán en una hash table como se mencionó con anterioridad, en donde la clave del mismo serán los atributos del query y en donde los valores de las misma serán los valores del query del URI.
\item fragment: Es una variable de tipo ``string'' que almacenará el fragment del URI en caso de poseerlo.
\item número de estados: es una variable de tipo entero que almacena el número de estados del autómata que se emplea en el módelo teórico fueron visitados para reconocer el URI. Este campo del registro será utilizado por el módulo de evaluación.
\item uri correcto: es una variable de tipo booleano que indica si el URI está sintácticamente correcto o no.
\end{itemize}

El nombre que recibirá este tipo de estructura es ``UriSegmentado''. La misma se puede apreciar gráficamente en la figura \ref{fig:uriSegmentado}.

\begin{figure}[!htb]
\begin{center}
\includegraphics[width=4in]{./img/UriSegmentado.jpg}
\caption{Ejemplo gráfico de la estructura ``UriSegmentado''.}
\label{fig:uriSegmentado}
\end{center}
\end{figure}

Además de utilizar el tipo de dato ``UriSegmentado'', el módulo de segmentación hace uso de una estructura de datos compuesta, primitiva del lenguaje de ``scripting'' de Bro llamada  ``URI''. Dicha estructura posee los siguientes campos:

\begin{itemize}
\item scheme: Es un campo opcional de tipo ``string'' que almacena el protocolo del URI.

\item netlocation: Es un campo de tipo ``string'' que almacena el nombre de dominio o la dirección IP del URI.

\item portnum: Es un campo opcional de tipo ``count''  que almacena el número de puerto del URI.
\item path:Es un campo de tipo ``string'' que almacena la ruta del URI.

\item file\_name: En un campo opcional de tipo ``string'' que almacena el nombre completo  del archivo de la ruta del URI junto a su extensión.

\item file\_base: En un campo opcional de tipo ``string'' que almacena el nombre  del archivo de la ruta del URI sin su extensión.

\item file\_ext: En un campo opcional de tipo ``string'' que almacena la extensión del archivo de la ruta.

\item params: Es un campo opcional de tipo ``table'' que almacena todos los parámetros de la consulta del URI. Esta tabla mapea todos los atributos con sus valores.

\end{itemize}

La estructura ``URI'' se puede apreciar gráficamente en la figura \ref{fig:URI}.

\begin{figure}[!htb]
\begin{center}
\includegraphics[width=4in]{./img/URIstruct.jpg}
\caption{Ejemplo gráfico de la estructura ``URI''.}
\label{fig:URI}
\end{center}
\end{figure}	

\section*{Estructura de Datos del Módulo de Evaluación}

El módulo de evaluación del sistema cuenta con varias estructura de datos compuestas de tipo ``register'' y de tipo ``table'' que serán explicados a continuación 

\subsection*{Modelo de normalidad}
\label{sssec:estructuraModelo}

Existen dos estructuras de tipo ``register'' en la función para leer el modelo de normalidad y dos de tipo ``table''.
Las estructuras de tipo ``register'' poseen los siguientes campo:

La primera, cuyo nombre es ``Word'' posee los siguientes campo:
\begin{itemize}
\item ``word'' es un campo de tipo ``string'' en donde se almacenarán las palabras del vocabulario de cada uno de los estados del autómata presentado en la imagen \ref{fig:ssm}.
\item ``state'' es un campo de tipo string en donde se indicará el estado del autómata presentado en la imagen \ref{fig:ssm} al que pertenece la palabra almacenada en el campo ``word''.
\end{itemize}

Esta estructura se puede observar de manera gráfica en la figura \ref{fig:WORD}.

\begin{figure}[!htb]
\begin{center}
\includegraphics[width=4in]{./img/Word.jpg}
\caption{Ejemplo gráfico de la estructura ``Word''.}
\label{fig:WORD}
\end{center}
\end{figure}	

La segunda estructura llamada ``Probability'' solo posee un campo llamado ``probability''.
\begin{itemize}
\item ``probability'' es un campo de tipo ``double'' en donde se almacenará la probabilidad de generación de las palabras que se encuentran en el modelo.
\end{itemize}

Se puede observar un ejemplo gráfico de la estructura en la figura \ref{fig:Probability}.

\begin{figure}[!htb]
\begin{center}
\includegraphics[width=4in]{./img/Probability.jpg}
\caption{Ejemplo gráfico de la estructura ``Probability''.}
\label{fig:Probability}
\end{center}
\end{figure}	

La tercera estructura de tipo ``register'' que será utilizada en el módulo de evaluación se llamada ``TableDescription'' y pertenece al módulo ``Input'' de Bro y se utilizará al momento de leer los archivos de entrada.
Los campos utilizados de esta estructura fueron los siguientes:
\begin{itemize}
\item source: campo de tipo ``string'' que almacena el nombre del archivo
que va a ser leído.
\item name: campo de tipo ``string'' que almacenará el nombre que se le
asignará al flujo de entrada.
\item destination: Nombre de la tabla que almacena la información contenida
en los archivos.
\item idx: Nombre del registro que definirá los valores que utilizará la tabla que almacena la información del archivo como clave.
\item val: Campo opcional que almacena el nombre del registro que define
los valores de la tabla que almacena la información de los archivos de
entrada.
\end{itemize}

En la figura \ref{fig:TableDescription} se muestra un ejemplo gráfico de ``TableDescription''.

\begin{figure}[!htb]
\begin{center}
\includegraphics[width=4in]{./img/TableDescription.jpg}
\caption{Ejemplo gráfico de la estructura ``TableDescription''.}
\label{fig:TableDescription}
\end{center}
\end{figure}	

Por otra parte, las estructuras de tipo ``table'' son dos. Una tabla será utilizada para almacenar las palabras del vocabulario, la probabilidad de generación y el nombre del estado al que pertenece dicha información mientras que la otra almacenará las probabilidades de fuera de vocabulario de cada uno de los estados y el valor del parámetro $\theta$ presente en la expresión \ref{eq:ClaseU}.

La primera tabla es una ``hash table'' que posee dos elementos tipo ``string'' como clave, y como valor tiene un campo de tipo ``Probability''.

Esta tabla almacenará el modelo de normalidad del sistema, es decir, las palabras del vocabulario, la probabilidad de generación de las mismas y el estado al que pertenece dicha información. Las claves de esta tabla serán las palabras del vocabulario y el nombre del estado al que pertenecen. Toda esta informacion será dada a través de un archivo de texto cuya estructura será explicada en las sección ~\ref{sec:lecturaModelo}.

La segunda tabla es una hash table que tendrá solo un elemento como clave de tipo ``string''. Los valores de la misma serán campo tipo ``Valor''. 

En dicha tabla se van a almacenar las probabilidades de fuera de vocabulario de cada uno de los estados del autómata y el parámetro $\theta$. La clave de la misma sería una cadena de caracteres que identifique cada una de las probabilidades  y el parámetro  $\theta$. Esas etiquetas serían las siguientes: Poov1, Poov2, Poov3, Poov4, Theta.  

\subsection*{Evaluación de las probabilidades de los URI}
\label{sssec:estructuraEvaluacion}

La función  que se encarga de calcular el índice de anormalidad  y evaluar si el mismo es anómalo o no, solo contiene una estructura compuesta de tipo ``record'' y se utiliza junto a una herramienta de Bro que funciona para escribir ``logs''. Los ``logs'' que son escritos a través de esta herramienta son archivos de texto que contiene una lista de los URI que presentan anomalías.

La estructura lleva como nombre ``InfoAtaque'' y contiene los siguientes campos:

\begin{itemize}
\item clasificacion: es un campo de tipo ``string'' que almacena la clasificación de los URIs anómalos, es decir, aquí se indica si el mismo presenta anomalía por estar sintácticamente mal construidos o porque el índice de anormalidad sobrepasó el parámetro $\theta$.
\item uri: es un campo de tipo ``string'' en el que se va a almacenar el uri que va a ser registrado en el log.
\item probability: es un campo de tipo ``string'' en donde se va a almacenar el valor del índice de anormalidad del uri.
\end{itemize}

Todos los campos anteriormente descritos poseen la cualidad de ser de tipo ``log'' tambien. Con esto se indica cuál de los elementos de la estructura de datos se escribirá sobre los logs.

    El registro ``InfoAtaque'' se puede observar de manera gráfica en la figura \ref{fig:InfoAtaque}. 
    
\begin{figure}[!htb]
\begin{center}
\includegraphics[width=4in]{./img/InfoAtaque.jpg}
\caption{Ejemplo gráfico de la estructura ``InfoAtaque''.}
\label{fig:InfoAtaque}
\end{center}
\end{figure}	

\section*{Estructura de Datos del Módulo de Entrenamiento}
El módulo de entrenamiento consta de varias estructuras de datos de
tipo ``register'' y de tipo ``table'' para realizar su trabajo. Tanto el modo
``Online'' como el ``Offline'' comparten las mismas estructuras.
La estructuras serían las siguientes:
``Entrenamiento'' es una estructura de tipo ``record'', cuyo funcionamiento es ir almacenando el número de veces que una palabra es observada junto con la probabilidad de observación, mientras se realiza el entrenamiento. Los
campos de esta estructura son los siguientes:

\begin{itemize}
\item numPalabras: campo de tipo entero que almacena el número de veces
que una palabra es observada durante el entrenamiento.
\item probability: campo de tipo flotante que almacene la probabilidad de observación de una palabra.
\end{itemize}

Un ejemplo gráfico de la estructura ``Entrenamiento'' se puede apreciar
en la figura \ref{fig:figEntrenamiento}.
La otra estructura de tipo ``record'' es ``Info''. ``Info'' se encargará de
almacenar la información que será escrita en el archivo de texto que representará el modelo de normalidad. Los campos de esta estructura de datos
son los siguientes:

\begin{figure}[!htb]
\begin{center}
\includegraphics[width=4in]{./img/EntrenamientoRegister.jpg}
\caption{Ejemplo gráfico de la estructura ``Entrenamiento''.}
\label{fig:figEntrenamiento}
\end{center}
\end{figure}	

\begin{itemize}
\item state: es un campo de tipo ``string'' que almacenará el nombre del
estado al que pertenece la palabra y su probabilidad de observación.
\item word: es un campo de tipo ``string'' que almacenará las palabras observadas durante el entrenamiento.
\item probability: es un campo de tipo flotante que almacenará la probabilidad de observación de las palabras.
\end{itemize}

La estructura ``Info'' se puede observar de manera gráfica en la figura
\ref{fig:figInfo}.

\begin{figure}[!htb]
\begin{center}
\includegraphics[width=4in]{./img/Info.jpg}
\caption{Ejemplo gráfico de la estructura ``Entrenamiento''.}
\label{fig:figInfo}
\end{center}
\end{figure}

Por otra parte, la estructura tipo ``table'' que se utilizó en la implementación de este módulo fueron las siguientes:
Se hizo uso de una tabla de hash que posee como clave un campo tipo
``string'' y como valor una estructura de datos de tipo ``Entrenamiento''
(fig. \ref{fig:figEntrenamiento}). Esta tabla tendrá como función almacenar las palabras que van apareciendo durante el entrenamiento, el número de veces que fueron observadas las mismas y la probabilidad de aparición . La clave de esta tabla serán las palabras observadas y el resto de la informacion sería almacenado en la estructura de datos ``Entrenamiento''.
Cada estado del autómata tendrá su tabla de entrenamiento propia para
de esta manera tener una mayor organización de las observaciones realizadas durante el entrenamiento.

\chapter{Apéndice B: Pruebas Funcionales Realizadas}
\label{apendiceB}
\lhead{Apéndice B. \emph{Pruebas Funcionales Realizadas}}

\subsection{Pruebas funcionales}

En esta sección se explicarán las pruebas funcionales aplicadas a cada
una de la funciones construidas en el sistema.

\subsubsection{Filtro HTTP/GET}

Para probar el filtro de tipo HTTP/GET se tomó la base de datos,
\textbf{db1.pcap}, que contiene tanto paquetes HTTP/GET, como paquetes correspondientes a la sesión HTTP. La prueba consistió en pasarle a la función construida la base de datos de paquetes para que esta imprimiera los URIs de las peticiones HTTP/GET que estuviesen contenidas en la misma. En la
figura \ref{fig:filtroHTTP} se muestra un esquema de la prueba que se realizó.

\begin{figure}[!htb]
\begin{center}
\includegraphics[width=2in]{./img/filtroHTTP.png}
\caption{Esquema de pruebas realizadas al filtro HTTP/GET.}
\label{fig:filtroHTTP}
\end{center}
\end{figure}

\subsubsection{Lectura de archivos}

La lectura de los archivos es de suma importancia, ya que el sistema utiliza un modelo de normalidad y un conjunto de parámetros que se introducen
al mismo a tráves de archivos.
La manera en la que se probó esta función consistió en leer ambos archivos de textos necesarios por el sistema: ``config'' y ``modeloBro.log'' y luego se imprimieron las tablas resultantes de la lectura de los mismos.
En la figura \ref{fig:configFile}, se puede observar el archivo ``config'' que leerá la función de lectura del sistema y en la figura \ref{fig:configResult}, se puede apreciar la forma en la
que la misma almacenó los datos en la tabla correspondiente.
Por otra parte, en la figura \ref{fig:modelFile}, está el archivo ``modeloBro.log'' que fué utilizado en la prueba realizada, mientras que en la figura \ref{fig:modelResult} se muestra el resultado arrojado por la misma.

\begin{figure}[!htb]
\begin{center}
\includegraphics[width=3in]{./img/configFile.jpg}
\caption{Archivo ``config''.}
\label{fig:configFile}
\end{center}
\end{figure}

\begin{figure}[!htb]
\begin{center}
\includegraphics[width=3in]{./img/tablaConfig.png}
\caption{Tabla que contiene información del archivo ``config''.}
\label{fig:configResult}
\end{center}
\end{figure}

\begin{figure}[!htb]
\begin{center}
\includegraphics[width=3in]{./img/modeloOffline.jpg}
\caption{Archivo ``modeloBro.log''.}
\label{fig:modelFile}
\end{center}
\end{figure}

\begin{figure}[!htb]
\begin{center}
\includegraphics[width=3in]{./img/tablaModeloBro.png}
\caption{Tabla que contiene información del archivo ``modeloBro.log''.}
\label{fig:modelResult}
\end{center}
\end{figure}

\subsubsection{Módulo de segmentación}

Como se mencionó en los secciones anteriores, el modulo de segmentación
cuenta con dos funciones principales: la función de normalización y la de
segmentación. A continuación se mostrará la manera en la que se probó cada una de las funciones.

\subsubsection*{Función de normalización}

Las pruebas realizadas a la función de normalización consistió en realizar una lista de URIs sin normalizar, ingresarlos como parámetros a la función y observar los resultados arrojados por la misma.
La lista de URIs utilizados en esta prueba se pueden observar en la figura \ref{fig:uriSinNorm}. Por otra parte, los resultados arrojados por la misma se muestran en la
figura \ref{fig:uriNorm}.

\begin{figure}[!htb]
\begin{center}
\includegraphics[width=3in]{./img/uriSinNorm.png}
\caption{URIs sin normalizar.}
\label{fig:uriSinNorm}
\end{center}
\end{figure}

\begin{figure}[!htb]
\begin{center}
\includegraphics[width=3in]{./img/uriNorm.png}
\caption{URIs normalizados.}
\label{fig:uriNorm}
\end{center}
\end{figure}


\subsubsection*{Función de segmentación}

La pruebas realizadas a la función de segmentación, consistieron en ingresarle un conjunto de URIs a la misma y observar la manera en la que
esta los segmentaba.

\subsubsection{Módulo de entrenamiento}
\subsubsection*{Modo ``Offline''}
Como se pudo observar en los secciones anteriores, el módulo de entrenamiento esta conformado por una función, ``entrenarOffline'', que que se
encarga de hacer la llamada a función de: ``entrenamientoOffline'', ``evaluarProbabilidad'', ``escribirArchivoOffline''.
Para probar la función, ``entrenamientoOffline'', se construyeron una serie de datos de tipo ``UriSegmentado'' (fig. \ref{fig:uriSegmentado}), provenientes de la base de datos \textbf{db3.pcap} y le fueron pasados como parámetro de entrada a la misma. Una vez procesados los datos, se observó la salida de la misma para verificar la correctitud.

Por otra parte, para probar la función ``evaluarProbabilidad'', se tomaron los resultados arrojados por ``entrenamientoOffline'' y se
observó el resultado que arrojaba la misma. 

La prueba que se le realizó a ``escribirArchivoOffline'', consistió en tomar el resultado arrojado por ``evaluarProbabilidad'' y observar el
archivo de salida que escribía dicha función. 

\subsubsection*{Modo ``Online''}
Al igual que el modo ``Offline'', el modo ``Online'' esta conformado por
una función,``entrenarOnline'', que se encarga de hacer la llamada a función
del resto:``entrenamientoOnline'',``escribirArchivoOnline''
Las pruebas de ``entrenamientoOnline'' consistieron en darle un modelo de normalidad y un conjunto de estructura
de tipo ``UriSegmentado'' y observar la salida que arrojaba la misma.

Para probar la función ``escribirArchivoOnline'', se tomó el resultado arrojado por ``entrenamientoOnline'' y se observó el archivo de salida escrito por la misma.
\subsubsection{Módulo de evaluación}

El modulo de evaluación consta de una función general, llamada ``evaluacion'' que se encarga de hacer la llamada a función de: ``epsiloSumatoria'',
``calcularIndiceAnormalidad'', ``evaluarIndiceAnormalidad'' y ``escribirReporte'', que a su vez se encargan de realizar el trabajo del módulo de evaluación.
Para probar la función ``epsiloSumatoria'', se ingresó un conjunto de datos de tipo de tipo ``UriSegmentado'', un modelo de normalidad previamente leído, un valor para el epsilon, y un conjunto de estado del autómata para observar los resultados que arrojaba la misma.
Para la prueba de la función ``calcularIndiceAnormalidad'', se tomaron
los resultados proporcionados por la función ``epsiloSumatoria'' y se observará el resultado obtenido.

Tanto los resultados obtenidos por ``epsiloSumatoria'' por ``calcularIndiceAnormalidad'' fueron verificados calculando de manera manual los valores
que estas funciones calculan, haciendo uso de las expresiones correspondientes.
Las pruebas de la función ``evaluarIndiceAnormalidad'' consistieron en
darle un  índice de anormalidad y un valor del parámetro $\theta$, como los que
se muestran en la figura \ref{fig:archivoConfig} y verificar si la misma clasificaba de manera adecuada.
Finalmente, para realizar la prueba de ``escribirReporte'' se le pasó una serie de URIs pertenecientes a la base de datos \textbf{db1.pcap}. Una vez realizado
esto, se observó el archivo de salida escrito por dicha función.

\addtocontents{toc}{\vspace{2em}}
\backmatter

\end{document}
